\documentclass[a4paper,10pt,ngerman]{scrartcl}
\usepackage{babel}
% \usepackage[T1]{fontenc}
\usepackage[utf8x]{inputenc}
\usepackage[a4paper,margin=2.5cm]{geometry}
\usepackage{todonotes}

% Die nächsten drei Felder bitte anpassen:
\newcommand{\Name}{Richard Wohlbold} % Teamname oder eigenen Namen angeben
\newcommand{\Einsendenummer}{00487}
\newcommand{\Aufgabe}{Aufgabe 2: Geburtstag}

% Kopf- und Fußzeilen
\usepackage{scrlayer-scrpage}
\setkomafont{pageheadfoot}{\textrm}
\ifoot{\Name}
\cfoot{\thepage}
\chead{\Aufgabe}
\ofoot{Team-ID: \Einsendenummer}

% Für mathematische Befehle und Symbole
\usepackage{amsmath}
\usepackage{amssymb}
\usepackage{icomma}

% Für Bilder
\usepackage{graphicx}
\usepackage{svg}

% Für Algorithmen
\usepackage{algpseudocode}
\usepackage{algorithm}

% Für Quelltext
\usepackage{listings}
\usepackage{color}
\definecolor{mygreen}{rgb}{0,0.6,0}
\definecolor{mygray}{rgb}{0.5,0.5,0.5}
\definecolor{mymauve}{rgb}{0.58,0,0.82}
\lstset{
  keywordstyle=\color{blue},commentstyle=\color{mygreen},
  stringstyle=\color{mymauve},rulecolor=\color{black},
  basicstyle=\footnotesize\ttfamily,numberstyle=\tiny\color{mygray},
  captionpos=b, % sets the caption-position to bottom
  keepspaces=true, % keeps spaces in text
  numbers=left, numbersep=5pt, showspaces=false,showstringspaces=true,
  showtabs=false, stepnumber=2, tabsize=2, title=\lstname,
  breaklines=true,
  postbreak=\mbox{\textcolor{red}{$\hookrightarrow$}\space},
}
\lstdefinelanguage{JavaScript}{ % JavaScript ist als einzige Sprache noch nicht vordefiniert
  keywords={break, case, catch, continue, debugger, default, delete, do, else, finally, for, function, if, in, instanceof, new, return, switch, this, throw, try, typeof, var, void, while, with},
  morecomment=[l]{//},
  morecomment=[s]{/*}{*/},
  morestring=[b]',
  morestring=[b]",
  sensitive=true
}

\lstset{literate=%
    {_}{{\_}}1
    {Ö}{{\"O}}1
    {Ä}{{\"A}}1
    {Ü}{{\"U}}1
    {ß}{{\ss}}1
    {ü}{{\"u}}1
    {ä}{{\"a}}1
    {ö}{{\"o}}1
    {~}{{\textasciitilde}}1
    {“}{{"}}1
    {„}{{"}}1
}

% Diese beiden Pakete müssen als letztes geladen werden
%\usepackage{hyperref} % Anklickbare Links im Dokument
\usepackage{cleveref}

% Daten für die Titelseite
\title{\Aufgabe}
\author{\Name\\Team-ID: \Einsendenummer}
\date{\today}



\begin{document}

\maketitle
\setcounter{tocdepth}{2}
\tableofcontents

\section{Lösungsidee}
\label{sec:idea}
\subsection{Generierung der Tabelle}
\label{sec:idea:generate}
Meine Lösungsidee für das Problem besteht darin, alle möglichen Terme, die durch die gegebenen Rechenoperationen aus der gegebenen Ziffer erhalten werden können, in einer Tabelle zu speichern.
Dabei wird nach der Anzahl der vorkommenden Ziffern verfahren:
Angefangen wird bei $n=1$ Ziffern.
Für $n=1$ Ziffern lässt sich ohne Berücksichtigung der Fakultätsfunktion nur ein Term finden, nämlich die Ziffer selbst.
Für $ n \geq 2$ Ziffern werden Terme mit einer geringeren Anzahlen an Ziffern $i,j$ über die gegebenen Grundrechenarten kombiniert, sodass $n=i+j$.
Dies funktioniert, da durch bei der Kombination zweier Terme keine neue Ziffer verwendet werden muss.
Bei der Kombination sollen so wenig Möglichkeiten, wie für die Vollständigkeit nötig, ausprobiert werden: die Terme werden bei den kommutativen Grundrechenarten ($+$, $*$) beispielsweise nicht vertauscht, da dort dasselbe Ergebnis entsteht, bei den nicht-kommutativen Grundrechenarten jedoch schon, da oft verschiedene Ergebnisse auftreten.

Um beispielsweise alle Terme für $n=5$ zu finden, werden alle Terme mit $i=1$ und $j=4$ und mit $i=2$ und $j=3$ über das kartesische Produkt kombiniert und für alle Rechenarten und Reihenfolgen gespeichert.

Es gibt vier Grundrechenarten, von denen zwei nicht-kommutativ sind.
Man kann für die Abschätzung der Anzahl der Ergebnisse jedoch auch die Operation der Division als kommutativ betrachten, da für $a,b \in \mathbb{Z}, a \neq b$ nur $\frac{a}{b} \in \mathbb{Z}$ oder (exklusiv) $\frac{b}{a} \in \mathbb{Z}$.
Für $a = b$ ist jedoch $\frac{a}{b} = \frac{b}{a}$, sodass die Division auch in diesem Fall als kommutativ betrachtet werden kann.
Dadurch ergeben sich fünf mögliche neue Terme durch jedes Termpaar, davon ausgehend, dass keine Dopplungen auftreten und alle Terme valide sind (z.B. keine Division durch 0).
Zusätzlich ergibt sich für jedes $n$ ein neuer Term, der aus $n$-Mal der Ziffer besteht, z.B. 555555.
$N(n)$ steht dabei für die Anzahl an Termen mit $n$ Ziffern:

\begin{align*}
N(1) &= 1 \\
N(n) &= \sum_{i=1}^{\left \lfloor \dfrac{n}{2} \right \rfloor}5*N(i)*N(n-i)+1 \\
\end{align*}

Es ergeben sich die $N$, die in Abbildung \ref{fig:anzahl-terme} zu sehen sind.

\begin{figure}[]
  \centering
  \begin{tabular}{|r|r|}
  $n$ & $N(n)$ \\ \hline
  1 & 1 \\ \hline
  2 & 6 \\ \hline
  3 & 31 \\ \hline
  4 & 337 \\ \hline
  5 & 2.617 \\ \hline
  6 & 28.003 \\ \hline
  7 & 270.763 \\ \hline
  8 & 3.167.389 \\ \hline
  9 & 32.709.949 \\ \hline
  10 & 381.968.185 \\ \hline
  11 & 4.204.739.605 \\ \hline
  12 & 50.353.610.041 \\ \hline
  13 & 571.587.737.611 \\ \hline
  14 & 6.901.952.684.137 \\ \hline
  15 & 80.413.187.288.797
  \end{tabular}
  \caption{Anzahl der möglichen Terme $N$ nach der Anzahl der vorkommenden Ziffern $n$, Worst-Case}
  \label{fig:anzahl-terme}
\end{figure}

Es ist zu erkennen, dass die Anzahl an möglichen Termen in dieser Abschätzung etwa um den Faktor 10 jeden Schritt wächst.
Man muss jedoch auch sagen, dass in dieser Abschätzung Duplikate, also Zahlen, für die ein Term bereits bekannt ist, bereits eingeschlossen ist.
Für diese wird zumindest Zeit aufgewendet, sie tragen in den folgenden Iterationen jedoch nicht zum Wachstum bei.
Insgesamt wächst die Berechnungszeit und die für die Tabelle benötigte Speicherplatz somit exponentiell.

Eine Maßnahme, um den Algorithmus schneller zu gestalten, wäre, große Tabellen zu berechnen und diese auf der Festplatte zu speichern.
Da es bei dieser Aufgabe jedoch um das effiziente Verfahren geht, verfolge ich diesen Schritt nicht weiter in meiner Lösung.

Aus der Aufgabe geht nicht klar hervor, ob das unäre Minus (z.B. in $-5$) verwendet werden darf.
Da das Programm mit weniger implementierten Operationen schneller abläuft, habe ich es nicht implementiert, es würde jedoch eine sinnvolle Erweiterung des Problems darstellen.

\subsection{Scannen}
\label{sec:idea:scan}
Um die Laufzeit des Programms zu verbessern, wartet der Algorithmus nicht darauf, bis die Zahl in der Tabelle auftaucht, sondern überprüft, ob man zwei Terme aus der Tabelle durch eine Grundrechenart kombinieren kann, um auf das gewünschte Ergebnis zu kommen.
Dieses Verfahren nenne ich \textit{Scan}.
Dazu wird für jede Zahl $k$, für die ein Term in der Tabelle steht, ein Partner $p$ berechnet, mit dem das gewünschte Ergebnis $e$ erhalten werden kann.
Dafür gibt es verschiedene Möglichkeiten:

\begin{alignat*}{3}
  e &= j + p \quad &\Leftrightarrow& \quad p = e - k \\
  e &= j - p \quad &\Leftrightarrow& \quad p = k - e \\
  e &= p - j \quad &\Leftrightarrow& \quad p = k + e \\
  e &= p * j \quad &\Leftrightarrow& \quad p = \frac{e}{k} \\
  e &= p / j \quad &\Leftrightarrow& \quad p = e * k \\
  e &= j / p \quad &\Leftrightarrow& \quad p = \frac{k}{e}
\end{alignat*}

Für jedes $k$ werden nun alle möglichen Partner $p$ und es wird geschaut, ob $p$ Teil der Tabelle ist.
Zusätzlich überprüft der Scan, ob $e$ an sich in der Tabelle enthalten ist.
Ist dies der Fall, wird es einer Menge hinzugefügt.
Der Term mit der geringsten Anzahl an Ziffern der Menge ist ein mögliches Ergebnis.
Der \textit{Scan} hat den Vorteil, dass er im Gegensatz zur Generierung der Tabelle ein linearer Schritt ist.
Wird die Tabelle als Datensruktur mit Zugriffszeit $\mathcal{O}(1)$ implementiert, werden für jedes $k$ maximal 6 Tabellenzugriffe durchgeführt. 
Für hohe $n$ ist ein Scan also deutlich schneller, als die Generierung einer neuen Tabelle.

\subsection{Zusammenfügen}
Es wird immer abwechselnd eine Tabelle mit einer Ziffer mehr als zuvor generiert und ein \textit{Scan} ausgeführt.
Dies wird solange wiederholt, bis ein Scan ein Ergebnis mit $m$ Ziffern zurückgibt und eine Tabelle mit $m-2$ Ziffern vorliegt und ein Scan ausgeführt wurde.

An diesem Punkt wird die Lösung mit $m$ Ziffern verwendet, da eine Lösung mit $n=m-2$ Ziffern in der Tabelle stehen würde, der Scan eine Lösung mit $n=m-1$ Ziffern erkannt hätte (es müssen sowohl $p$ als auch $k$ bereits in der Tabelle vorhanden sein) und eine Lösung mit $n=m$ Ziffern bereits vorliegt.

Der Algorithmus kann nicht schon bei der Tabelle mit $m-3$ Ziffern terminieren, da eine Lösung, die sich aus Termen mit einer Ziffer und $m-2$ Ziffern zusammensetzt, durch einen \textit{Scan} nicht erkannt werden würde, sodass der Algorithmus nicht die beste Lösung liefern würde.
Algorithmus \ref{alg:combination} zeight die Kombination der Tabellengeneration und des Scans.

Eine Ausnahme stellen Zahlen da, die nur aus der gegebenen Ziffer oder dem Ergebnis ihrer Fakultätsfunktion dargestellt werden kann.
Da dem Scan nicht die Tabelle für $m-1$ Ziffern vorliegt, erkennt er nicht, ob die Zahl aus $m-1$ mal der gegebenen Ziffer oder ihrer Fakultätsfunktion gebildert werden kann.
In diesem Fall würde der Scan einen zu langen Term zurückgeben.
Um dagegen vorzugehen, muss im Schritt der Generierung aller Terme mit $n$ Ziffern bereits die Zahl mit $n+1$ mal der gegebenen Ziffern, sowie aller ihrer Fakultätsfunktionsergebnisse den ensprechenden Tabellen hinzugefügt werden.
So erkennt der anschließende Scan auch, ob es nicht zusammengesetzte Terme aus $m-1$ Ziffern gibt, die den Wert von $e$ annehmen.


\begin{algorithm}
  \caption{Kombinieren von \textsc{Generate} und \textsc{Scan}, um den Term mit der geringsten Anzahl an Ziffern zu finden}
  \begin{algorithmic}[1]
    \State $i \gets 1$
    \State $n \gets \infty$
    \State $r \gets$ nil
    \While{$i \leq n - 2$}
        \State \Call{Generate}{i}
        \State $r \gets$ \Call{Scan}{}
        \If{$r \neq$ nil}
          \State $n \gets$ \Call{Digits}{$r$}
        \EndIf
        \State $i \gets i+1$
    \EndWhile
    \State \textbf{return} r
  \end{algorithmic}
  \label{alg:combination}
\end{algorithm}

\subsection{Fakultät und Potenzieren}
Bei sowohl der Fakultätsfunktion als auch dem Potenzieren treten sehr große Zahlen auf, was das Programm extrem verlangsamen kann.
Deshalb wird vor beiden Operationen bestimmt, wie viele Stellen das Ergebnis hat.
Ist das Ergebnis zu lang, wird es nicht berechnet und auch nicht der Tabelle hinzugefügt.
Für die maximale Länge des Ergebnisses wähle ich 100 Stellen, da auch wissenschaftliche Taschenrechner mit Zahlen jenseits dieser Höhe nicht zurechtkommen.

\subsubsection{Potenzieren}
Über den Zehnerlogarithmus und die Logarithmusgesetze kann die Länge des Ergebnisses schnell berechnet werden:

\begin{align*}
  N &= a^b \\
  \Rightarrow \quad \mathrm{len}(N) &= \left \lfloor \log_{10}a^b \right \rfloor + 1\\
  \Leftrightarrow \quad \mathrm{len}(N) &= \left \lfloor b * \log_{10}a \right \rfloor + 1
\end{align*}

Da das Potenzieren eine nicht-kommutative binäre Operation ist, ähnelt sie den Grundrechenarten und kann in der Generierung der Tabelle hinzugefügt werden, sodass für jede Kombination zweier Terme der Tabelle 7 neue Terme hinzugefügt werden.

Dem \textit{Scan} werden auch zwei Prüfungen hinzugefügt, ob sich das Ergebnis $e$ durch einen Term der Tabelle $k$ und seinen Partner $p$ darstellen lassen:

\begin{alignat*}{3}
  e &= p ^ k \quad &\Leftrightarrow& \quad p = e ^ {\frac{1}{k}} \\
  e &= k ^ p \quad &\Leftrightarrow& \quad p = \log_k e
\end{alignat*}

\subsubsection{Fakultätsfunktion}
Aufgrund des schnellen Anstiegs der Fakultätsfunktion, kann die maximale Zahl, für die die Fa\-kul\-täts\-funk\-tion die gegebene maximale Anzahl an Stellen $l_{\mathrm{max}}$ nicht überschreitet, einfach berechnet werden.

Da die Fakultätsfunktion eine unäre Funktion ist, verlangt sie nach keinen weiteren Ziffern.
Ein Term und seine Fakultät haben deshalb dieselbe Anzahl an Stellen.
Beim Erstellen der Tabelle wird dann jeder Term, der der Tabelle hinzugefügt wird und einen Wert $\geq 3$ und $\leq l_{\mathrm{max}}$ aufweist, auch als Fakultät der Tabelle hinzugefügt, sofern diese noch nicht enthalten ist.
Dies wird rekursiv gelöst, damit beispielsweise auch $3!! = 720$ der Tabelle hinzugefügt wird.

\begin{algorithm}
  \caption{Prozedur, um der Tabelle einen Term hinzuzufügen, unter Berücksichtigung der Faktultätsfunktion}
  \begin{algorithmic}[1]
    \Procedure{AddToTable}{T}
      \If{$3 \leq $ \Call{Value}{t} $\leq l_{\mathrm{max}}$}
        \State $f \gets $ \Call{Factorial}{T}
        \State \Call{AddToTable}{f}
      \EndIf
      \State $d \gets$ \Call{Digits}{T}
      \If{$T \notin t$ $\land$ \Call{Digits}{T} $<$ \Call{Digits}{}}
        \State \Call{Add}{t, T}
      \EndIf
    \EndProcedure
  \end{algorithmic}
  
\end{algorithm}


\section{Umsetzung}
Ich habe die in Abschnitt \ref{sec:idea} geschilderte Lösungsidee in Python 3 umgesetzt.

\subsection{Repräsentation eines Terms}
Ich arbeite mit den objektorientierten Features von Python, um Terme als Datenstrukturen zu repräsentieren.

Dazu definiere ich eine Superklasse \texttt{Term}, die alle Methoden enthält, die Subklassen auch bereitstellen müssen.
Diese Methoden lösen bei \texttt{Term} einen \texttt{NotImplementedError} aus, sodass sofort auffällt, wenn eine Subklasse eine der Methoden nicht implementiert.
Die Methoden von \texttt{Term} sind die folgenden:
\begin{itemize}
  \item \texttt{number\_of\_digits}: Diese Methode gibt die Anzahl der im Term vorkommenden Ziffern zurück.
  \item \texttt{value}: Diese Methode gibt den numerischen Wert des gesamten Terms zurück.
  \item \texttt{\_\_str\_\_}: Diese Methode gibt den Term als Textrepräsentation zurück.
  \item \texttt{\_\_repr\_\_}: Diese Methode gibt im Fall von \texttt{Term} den Term auch den Term als Textrepräsentation zurück.
  Sie ist so defininiert, dass sie dasselbe Ergebnis wie \texttt{\_\_str\_\_} zurückgibt.
\end{itemize}

Nun gibt es verschiedene Varianten eines Terms.

Die einfachste Variante ist eine Zahl, die eine gewisse Ziffer ein- oder mehrmals enthält (un zwar nur diese).
Die Klasse heißt \texttt{Number} und erhält ihren Wert als Konstruktorargument.
Ihr numerischer Wert gleicht logischerweise ihrem inneren Wert.
Die Anzahl an Stellen lässt sich durch die Länge der Textrepräsentation berechnen.
Ihre Textrepräsentation ist die Zahl als \texttt{str}.

Die zweite Möglichkeit eines Terms ist eine unäre Operation, d.h. eine Operation, die auf einem Term operiert.
Im Fall der Aufgabenstellung ist dies nur die Fakultätsfunktion.
Die Klasse heißt \texttt{UnaryOperation} und erhält in ihrem Konstruktor den Term und die entsprechende Operation (nur \texttt{OP\_FAC} erlaubt).
Die Berechnung erfolgt, indem die Fakultätsfunktion für den Wert des inneren Terms berechnet wird.
Als Text wird die unäre Operation repräsentiert, indem der Term, gefolgt von einem \textit{!} in Klammern gesetzt wird.

Die dritte Möglichkeit eines Terms ist die binäre Operation, d.h. eine Operation die auf zwei Termen operiert.
Im Fall der Aufgabenstellung sind dies die vier Grundrechenarten sowie die Potenzfunktion.
Die Klasse heißt \texttt{BinaryOperation} und erhält in ihrem Konstruktor die beiden Terme und die entsprechende Operation (nur \texttt{OP\_ADD}, \texttt{OP\_SUB}, \texttt{OP\_MULT}, \texttt{OP\_DIV}, \texttt{OP\_POW} erlaubt).
Die Berechnung erfolgt, in dem die Operation auf die Werte der beiden inneren Terme ausgeführt wird.
Als Text wird die binäre Operation repräsentiert, indem die Terme, getrennt durch das Symbol der entsprechenden Operation, in Klammern gesetzt werden.

Durch das Einklammern aller zusammengesetzten Terme sorgt das Programm dafür, dass am Ende keine Probleme mit der Präzedenz der Operatoren auftreten.

\subsection{Datenstruktur der Tabelle}
Die Tabelle, die einer Zahl den Term mit der geringsten Anzahl an Ziffern zuordnet, stelle ich als \texttt{dict} dar.
Dabei ist der Schlüssel des \texttt{dict}s die Zahl selbst und der Wert ein Tupel aus einem \texttt{Term}-Objekt und der Anzahl an Ziffern im Term, die zwischengespeichert wird, um die Geschwindigkeit des Programms zu erhöhen.
Beispielhaft sieht die Tabelle für $n=1$ und $N=2$ mit Erlauben der Fakultätsfunktion folgendermaßen aus (die Repräsentation als Zeichenkette der Terme ist abgedruckt):

\begin{lstlisting}[language=Python]
{
  1: (1, 1),
  39916800: ((11!), 2),
  11: (11, 2),
  2: ((1+1), 2),
  0: ((1-1), 2)
}
\end{lstlisting}

\subsection{Generierung der Tabelle}
Ich definiere eine Funktion \texttt{add\_to\_table}, die einer Tabelle einen Term hinzufügt.
Dabei überprüft die Funktion, ob ein Term mit gleich vielen oder weniger Ziffern bereits in der Tabelle existiert und fügt ihn ihr sonst hinzu.
Zustätzlich kann ein Parameter übergeben werden (\texttt{extended}), der festlegt, ob die Funktion auch die Fakultät des Terms hinzufügen soll.
Um Berechnungen zu vermeiden, bei denen sehr lange Zahlen entstehen, wird die Fakultät eines Terms nur aufgenommen, falls sie 100 oder weniger Stellen besitzt.
Der höchste erlaubte Wert eines Terms, dessen Fakultät in die Tabelle aufgenommen wird, wird am Anfang des Programms berechnet und in \texttt{MAX\_FACTORIAL} gespeichert.
In diesem Prozess wird zusätzlich ein \texttt{dict} erstellt (\texttt{FACTORIALS}), das einem Ergebnis der Fakultätsfunktion seinem Parameter zuordnet.
Dieses \texttt{dict} wird später in Abschnitt \ref{sec:implementation:scan} verwendet.

Die Funktion \texttt{generate} generiert alle Terme der Tabelle, die \texttt{num\_digits} Ziffern enthalten.
Dabei differenziert die Funktion zwischen zwei Tabellen: Der \texttt{aggegated\_table} und der \texttt{split\_table}.

Die \texttt{aggregated\_table} enthält alle, bis zu diesem Zeitpunkt durch die Ziffer dargestellten Terme.
Sie folgt dem oben geschilderten Aufbau.
Sie wird in der Funktion \texttt{scan} eingesetzt, um den Partner $p$ zu ermitteln.

Zusätzlich gibt es die \texttt{split\_table}, die verschiedene Tabellen enthält.
Dabei ordnet sie einer Anzahl an Ziffern $N$ eine Tabelle zu, die nur Terme mit dieser Anzahl an Ziffern enthält.
Beispielhaft sieht die \texttt{split\_table} für $n=1$ und $N=2$ mit Erlauben der Fakultätsfunktion folgendermaßen aus (die Repräsentation als Zeichenkette der Terme ist abgedruckt):
\begin{lstlisting}[language=Python]
{
  1: {
    1: (1, 1)
  }, 
  2: {
    11: (11, 2),
    2: ((1+1), 2),
    0: ((1-1), 2),
    1: ((1/1), 2)
  }
}
\end{lstlisting}

Die Funktion geht nun nach dem in Abschnitt \ref{sec:idea} geschilderten Verfahren vor.
Sie variiert insgesamt die Anzahl an Ziffern des 1. Terms, aus der sich die Anzahl an Ziffern des 2. Terms ergibt.
Im Folgenden iteriert sie durch alle Terme mit den gegebenen Anzahlen an Ziffern (aus \texttt{split\_table}), sodass alle Kombinationen gefunden werden.
Die Terme werden nun nach ihrer Größe geordnet (für die Division) und mit allen möglichen Rechenarten (falls \texttt{extended} auch mit der Potenzfunktion) verbunden und jeweils mit \texttt{add\_to\_table} der Tabelle hinzugefügt.

Am Ende der Funktion werden alle neuen Terme der neu generierten \texttt{split\_table} der \texttt{aggregated\_table} hinzugefügt.


\subsection{Scan}
\label{sec:implementation:scan}
Die Funktion \texttt{scan} führt das in Abschnitt \ref{sec:idea:scan} geschilderte Verfahren aus.
Falls \texttt{extended} \texttt{True} ist, wird auch die Potenzfunktion und die Fakultätsfunktion umgekehrt verwendet.
Um die Fakultätsfunktion umzukehren, wird das eingangs generierte \texttt{dict} \texttt{FACTORIALS} verwendet, das als Wert die Zahl ausgibt, deren Fakultät den Schlüssel ergibt.
Alle gefundenen Möglichkeiten werden einem \texttt{set} hinzugefügt.
Am Ende der Funktion wird die Möglichkeit mit der geringsten Anzahl an Ziffern ausgewählt.
Falls zwei Möglichkeiten dieselbe Anzahl an Ziffern haben, wird die mit weniger Zeichen gewählt.
Falls eine Möglichkeit gefunden wurde, wird diese zurückgegeben.
Ansonsten wird \texttt{None} zurückgegeben.

\subsection{Zusammenfügen}
Die Funktion \texttt{find\_shortest} fügt nun \texttt{generate\_table} und \texttt{scan} zusammen.
Nach dem in Anbschnitt \ref{sec:idea} geschilderten Verfahren wird abwechselnd die Tabelle generiert und ein \texttt{Scan} durchgeführt, eine Tabelle mit $m-2$ Ziffern vorliegt, wobei $m$ die Anzahl an Ziffern des gefundenen Terms ist.
Diese Funktion wird insgesamt zweimal ausgeführt, einmal ohne und einmal mit \texttt{extended}.
Zusätzlich kann festgelegt werden, ob zusätzliche Informationen ausgegeben werden sollen (\texttt{debug}).

Das Programm benutzt einen \texttt{ArgumentParser}, um die Zahl und die Ziffer aus den Argumenten zu ermitteln, sowie zu überprüfen, ob eine zusätzliche Ausgabe erfolgen soll.

Sobald endgültige Terme im eingeschränkten bzw. erweiterten Modus gefunden wurden, werden diese mit der Anzahl an Stellen ausgegeben.

\subsection{Laufzeitanalyse}
Das Element des Programms, welches am längsten benötigt, ist die Generierung der Tabelle.
Die Zeit, die für diese genötigt wird, verhält sich exponentiell zu der Anzahl der Ziffern, für die die Funktion \texttt{generate} aufgerufen wird.
Dies lässt sich auf einem Graphen mit logarithmischer y-Skala, der die Generierungszeit der Tabellen gegen die Anzahl der Ziffern, mit der die Funktion aufgerufen wird, abbildet.
Ein solcher Graph ist beispielhaft für die Ziffer 4 in Abbildung \ref{fig:implementation:runtime:1} zu erkennen.

\begin{figure}
  \centering
  \includesvg{diagramm1.svg}
  \caption{Generierungszeit der Tabellen für die Ziffer 4 nach Anzahl der Ziffern pro Term}
  \label{fig:implementation:runtime:1}
\end{figure}

Um die Geschwindigkeit des Ansteigens zu berechnen, führe ich eine lineare Regression des natürlichen Logarithmus der Generierungszeit gegen die Anzahl der Ziffern pro Term für jede Ziffer durch.
Dabei entspricht die erhaltene Steigung $m$ der Wachstumskonstante $k$ beim exponentiellen Wachstum.
Die Wachstumskonstante rechne ich nun in den Wachstumsfaktor $a$ um.
Die Regression führe ich für sämtliche Ziffern je zweimal durch: Einmal ohne und einmal mit Fakultäts- und Potenzfunktion.

\begin{alignat*}{2}
  & & \ln{t(n)} &= k*n + b \\
  & & t(n) &= e^b * e^{k*n} = t_0 * e^{k*n} \\
  &\Leftrightarrow \quad& t(n) &= t_0 * (e^k)^n = t_0 * a^n \\
  &\Rightarrow \quad& a &= e^k
\end{alignat*}

Die Wachstumskonstante $a$ stelle ich in einem Säulendiagramm für jede Ziffer dar (siehe Abbildung \ref{fig:implementation:runtime:2}).

\begin{figure}
  \centering
  \includesvg{diagramm2.svg}
  \caption{Wachstumsfaktoren der Generierungszeiten nach Ziffern, ohne und mit Fakultäts- und Potenzfunktion}
  \label{fig:implementation:runtime:2}
\end{figure}

Aus beiden Abbildungen zusammen kann erkannt werden, dass die Laufzeit der Tabellengenerierung exponentiell zunimmt, dies aber unterschiedlich schnell geschieht.
Es können somit nur eine beschränkte Anzahl an Aussagen zur Laufzeit getroffen werden, da sich die Anzahl der benötigten Tabellen je nach Zahl und Ziffer stark unterscheiden können.
Die Generierung der Tabellen mit Fakultäts- und Potenzfunktion braucht aber immer deutlich länger, vor allem für Ziffern $\ge 3$.
So braucht die Tabelle mit Termen mit 7 Ziffern für die Ziffer 3 ca. $\frac{13^7}{3.8^7} \approx 5484$ Mal länger, wenn Fakultäts- und Potenzfunktionen miteinbezogen werden.
Insgesamt kann man aber auch erkennen, dass ohne Faktuläts- und Potenzfunktionen der Wachstumsfaktor der Generierungszeiten deutlich unter den abgeschätzten 10 (siehe \ref{sec:idea:generate}) liegt. 
Dies zeigt, dass ein großer Teil der generierten Terme nicht valide oder Duplikate sind und daher vermieden werden können.
Beispielhafte Laufzeiten lassen sich im Abschnitt \ref{sec:examples} erkennen.

\section{Beispiele}
\label{sec:examples}
Für jede Ziffer führe ich das Programm für die Zahlen 2019, 2030, 2080 und 2980 aus.
Jedes Beispiel rufe ich unter Linux mit dem Programm \texttt{time} auf, das am Ende die Zeit ausgibt, die das Programm gebraucht hat.
\subsection{Ziffer 1}
\subsubsection{2019}
Der von meinem Programm gefundene Term hat genauso viele Ziffern wie das Beispiel aus der Aufgabenstellung:
\begin{lstlisting}
$ command time 'time: %E' python main.py 2019 1
normal result ((((11111-1)/11)*(1+1))-1)
digits: 11

extended result (((1+1)^11)-(((11-1)*((1+1)+1))-1))
digits: 11
0:05.42
\end{lstlisting}
\subsubsection{2030}
\begin{lstlisting}
$ command time 'time: %E' python main.py 2030 1
normal result ((1-11)*(((1+1)*((11-1)-111))-1))
digits: 12

extended result (((1+1)^11)-((((1+1)+1)!)*((1+1)+1)))
digits: 10
0:01.45
\end{lstlisting}
\subsubsection{2080}
\begin{lstlisting}
$ command time 'time: %E' python main.py 2080 1
normal result ((11-1)*(((111-1)*(1+1))-(11+1)))
digits: 12

extended result (((1+1)^11)+((1+1)^((((1+1)+1)!)-1)))
digits: 10
0:01.43
\end{lstlisting}
\subsubsection{2980}
\begin{lstlisting}
$ command time 'time: %E' python main.py 2980 1
normal result ((1-11)*((1+1)+(((1+1)+1)*(11-111))))
digits: 13

extended result ((((((1+1)+1)!)+1)!)-(((1+1)^11)+(11+1)))
digits: 11
0:05.81
\end{lstlisting}
\subsection{Ziffer 2}
\subsubsection{2019}
Der von meinem Programm gefundene Term hat genauso viele Ziffern wie das Beispiel aus der Aufgabenstellung:
\begin{lstlisting}
$ command time 'time: %E' python main.py 2019 2
normal result ((((((22*2)+2)*22)-2)*2)-(2/2))
digits: 10

extended result ((((22*2)+(2/2))^2)-((2+(2/2))!))
digits: 9
0:02.42
\end{lstlisting}
\subsubsection{2030}
\begin{lstlisting}
$ command time 'time: %E' python main.py 2030 2
normal result (2+(((((22*2)+2)*22)+2)*2))
digits: 9

extended result (((2^(22/2))-22)+(2^2))
digits: 8
0:00.54
\end{lstlisting}
\subsubsection{2080}
\begin{lstlisting}
$ command time 'time: %E' python main.py 2080 2
normal result (((22+(2*2))*(2*2))*(22-2))
digits: 9

extended result (((22-2)*2)*((((2^2)!)+2)*2))
digits: 8
0:00.38
\end{lstlisting}
\subsubsection{2980}
\begin{lstlisting}
$ command time 'time: %E' python main.py 2980 2
normal result ((2-((22+2)*((2*(2-22))-22)))*2)
digits: 11

extended result (((2^(2^(2^2)))+((2^2)!))/22)
digits: 8
0:02.97
\end{lstlisting}
\subsection{Ziffer 3}
\subsubsection{2019}
Der von meinem Programm gefundene Term hat genauso viele Ziffern wie das Beispiel aus der Aufgabenstellung:
\begin{lstlisting}
$ command time 'time: %E' python main.py 2019 3
normal result (3+((333+3)*(3+3)))
digits: 7

extended result ((((3!)^3)*(3!))+(((3!)!)+3))
digits: 5
0:00.07
\end{lstlisting}
\subsubsection{2030}
\begin{lstlisting}
$ command time 'time: %E' python main.py 2030 3
normal result (33+((333*(3+3))-(3/3)))
digits: 9

extended result (((((3!)!)-3)*3)-((((3!)!)+(3!))/(3!)))
digits: 6
0:00.22
\end{lstlisting}
\subsubsection{2080}
\begin{lstlisting}
$ command time 'time: %E' python main.py 2080 3
normal result ((3/3)+((33*3)*(((3+3)*3)+3)))
digits: 9

extended result ((((3!)!)*3)-((((3!)!)/3)/3))
digits: 5
0:00.15
\end{lstlisting}
\subsubsection{2980}
\begin{lstlisting}
$ command time 'time: %E' python main.py 2980 3
normal result ((((333*3)-(3+3))*3)+(3/3))
digits: 9

extended result ((((3!)!)+(3/3))+((((3!)!)+33)*3))
digits: 7
0:01.49
\end{lstlisting}
\subsection{Ziffer 4}
\subsubsection{2019}
Der von meinem Programm gefundene Term hat genauso viele Ziffern wie das Beispiel aus der Aufgabenstellung:
\begin{lstlisting}
$ command time 'time: %E' python main.py 2019 4
normal result ((4-(4/4))+((((4*4)*(4*4))-4)*(4+4)))
digits: 10

extended result ((((4+4)!)/((4!)-4))+(4-(4/4)))
digits: 7
0:02.06
\end{lstlisting}
\subsubsection{2030}
\begin{lstlisting}
$ command time 'time: %E' python main.py 2030 4
normal result (((444+((4*4)*4))*4)-((4+4)/4))
digits: 10

extended result (((((4+4)!)+(4!))+(4^4))/((4!)-4))
digits: 7
0:02.09
\end{lstlisting}
\subsubsection{2080}
\begin{lstlisting}
$ command time 'time: %E' python main.py 2080 4
normal result ((((4*4)*(4*4))+4)*(4+4))
digits: 7

extended result (((4^4)+4)*(4+4))
digits: 5
0:00.05
\end{lstlisting}
\subsubsection{2980}
\begin{lstlisting}
$ command time 'time: %E' python main.py 2980 4
normal result (((((4*4)*4)+4)*44)-((4*4)-4))
digits: 9

extended result ((((((4!)/4)!)+(4!))*4)+4)
digits: 5
0:00.26
\end{lstlisting}
\subsection{Ziffer 5}
\subsubsection{2019}
Der von meinem Programm gefundene Term hat genauso viele Ziffern wie das Beispiel aus der Aufgabenstellung:
\begin{lstlisting}
$ command time 'time: %E' python main.py 2019 5
normal result ((((((55+(5*5))*5)+5)*5)-5)-(5/5))
digits: 10

extended result (((((5!)+5)*((5+5)+5))+(5!))+((5!)/5))
digits: 8
0:04.84
\end{lstlisting}
\subsubsection{2030}
\begin{lstlisting}
$ command time 'time: %E' python main.py 2030 5
normal result (5+((((55+(5*5))*5)+5)*5))
digits: 8

extended result (((5!)+(5*5))*(((5!)/5)-(5+5)))
digits: 7
0:00.45
\end{lstlisting}
\subsubsection{2080}
\begin{lstlisting}
$ command time 'time: %E' python main.py 2080 5
normal result ((((5*5)*5)+5)*((55/5)+5))
digits: 8

extended result ((5^5)-(55*(((5!)/5)-5)))
digits: 7
0:00.37
\end{lstlisting}
\subsubsection{2980}
\begin{lstlisting}
$ command time 'time: %E' python main.py 2980 5
normal result ((55*(55-(5/5)))+(5+5))
digits: 8

extended result (((5^5)-(5!))-(5*5))
digits: 5
0:00.07
\end{lstlisting}
\subsection{Ziffer 6}
\subsubsection{2019}
Der von meinem Programm gefundene Term hat genauso viele Ziffern wie das Beispiel aus der Aufgabenstellung:
\begin{lstlisting}
$ command time 'time: %E' python main.py 2019 6
normal result (((((666+6)*6)+6)*6)/(6+6))
digits: 9

extended result (((6!)+(6!))-(((((6!)+6)+((6!)/6))/6)-(6!)))
digits: 8
0:03.06
\end{lstlisting}
\subsubsection{2030}
\begin{lstlisting}
$ command time 'time: %E' python main.py 2030 6
normal result (((6/6)-(6*6))*(((6+6)/6)+(6-66)))
digits: 10

extended result ((((6^6)+(6!))/(6*6))-(6-(6!)))
digits: 7
0:00.78
\end{lstlisting}
\subsubsection{2080}
\begin{lstlisting}
$ command time 'time: %E' python main.py 2080 6
normal result (((6-(6*6))-((6+6)/6))*((6/6)-66))
digits: 10

extended result ((((6!)+(6!))+(6!))-(((6!)+(6!))/((6+6)+6)))
digits: 8
0:03.39
\end{lstlisting}
\subsubsection{2980}
\begin{lstlisting}
$ command time 'time: %E' python main.py 2980 6
normal result (((((6*6)*(6+6))-6)*(6+(6/6)))-((6+6)/6))
digits: 11

extended result (((((6!)-((6!)/6))/6)-((6!)+(6!)))+((6!)*6))
digits: 8
0:05.09
\end{lstlisting}
\subsection{Ziffer 7}
\subsubsection{2019}
Der von meinem Programm gefundene Term hat genauso viele Ziffern wie das Beispiel aus der Aufgabenstellung:
\begin{lstlisting}
$ command time 'time: %E' python main.py 2019 7
normal result (((77-7)/7)-(7*(7+(7*(7-(7*7))))))
digits: 10

extended result (((((7!)+(7!))*((7!)+7))+(7!))/(((7!)*7)-((7!)+(7!))))
digits: 9
0:36.72
\end{lstlisting}
\subsubsection{2030}
\begin{lstlisting}
$ command time 'time: %E' python main.py 2030 7
normal result (((7-(7*(7-(7*7))))*7)-77)
digits: 8

extended result (7*(((7!)/(7+7))+(7-77)))
digits: 7
0:04.35
\end{lstlisting}
\subsubsection{2080}
\begin{lstlisting}
$ command time 'time: %E' python main.py 2080 7
normal result ((7-(77*(7-((7+7)*(7+7)))))/7)
digits: 9

extended result ((((((7*7)+7)*7)+7)*7)-(((7!)/7)-7))
digits: 9
0:31.23
\end{lstlisting}
\subsubsection{2980}
\begin{lstlisting}
$ command time 'time: %E' python main.py 2980 7
normal result ((((77+7)*7)+(7+(7/7)))*(7-((7+7)/7)))
digits: 11

extended result (((7!)-((7+7)/7))-((7*7)*((7*7)-7)))
digits: 9
0:33.10
\end{lstlisting}
\subsection{Ziffer 8}
\subsubsection{2019}
Der von meinem Programm gefundene Term hat genauso viele Ziffern wie das Beispiel aus der Aufgabenstellung:
\begin{lstlisting}
$ command time 'time: %E' python main.py 2019 8
normal result ((88+(88/8))-(((8+8)+8)*(8-88)))
digits: 11

extended result (((((((8!)+(8!))+(8!))/8)+8)/8)+((8+8)*8))
digits: 9
0:31.39
\end{lstlisting}
\subsubsection{2030}
\begin{lstlisting}
$ command time 'time: %E' python main.py 2030 8
normal result (((8+8)/8)*(((8*8)*(8+8))-(8+(8/8))))
digits: 10

extended result ((((8!)/8)/8)-(8-(88*(8+8))))
digits: 8
0:03.18
\end{lstlisting}
\subsubsection{2080}
\begin{lstlisting}
$ command time 'time: %E' python main.py 2080 8
normal result (((((8*8)*8)+8)*(8*8))/(8+8))
digits: 8

extended result (((888*8)+8)+(8-((8!)/8)))
digits: 8
0:02.60
\end{lstlisting}
\subsubsection{2980}
\begin{lstlisting}
$ command time 'time: %E' python main.py 2980 8
normal result (((((8*8)*8)*((8*8)*8))+(88+8))/88)
digits: 11

extended result ((8/8)+((((8^8)/(8*8))+8)/88))
digits: 9
0:30.48
\end{lstlisting}
\subsection{Ziffer 9}
\subsubsection{2019}
Der von meinem Programm gefundene Term hat genauso viele Ziffern wie das Beispiel aus der Aufgabenstellung:
\begin{lstlisting}
$ command time 'time: %E' python main.py 2019 9
normal result (9+(((999*(9+9))+(99+9))/9))
digits: 10

extended result (99-((9!)/((9-99)-99)))
digits: 8
0:04.34
\end{lstlisting}
\subsubsection{2030}
\begin{lstlisting}
$ command time 'time: %E' python main.py 2030 9
normal result (((9+9)+(99/9))*((9*9)-(99/9)))
digits: 10

extended result ((((9+9)/9)^(99/9))-(9+9))
digits: 8
0:02.57
\end{lstlisting}
\subsubsection{2080}
\begin{lstlisting}
$ command time 'time: %E' python main.py 2080 9
normal result ((((9+9)+9)-(9/9))*((9*9)-(9/9)))
digits: 9

extended result (((9*9)*(9*9))-((((9!)/9)+9)/9))
digits: 8
0:02.11
\end{lstlisting}
\subsubsection{2980}
\begin{lstlisting}
$ command time 'time: %E' python main.py 2980 9
normal result ((((((9+9)*(9+9))+9)*9)+(9/9))-(9+9))
digits: 10

extended result (((((9!)/(9+9))+99)/9)+((9*9)*9))
digits: 9
0:21.87
\end{lstlisting}


\subsection{Nicht genügend Parameter}
Falls dem Programm nicht genügend Parameter übergeben werden, gibt es einen Fehler aus:
\begin{lstlisting}
$ python main.py 
usage: main.py [-h] [--verbose] number digit
main.py: error: the following arguments are required: number, digit
\end{lstlisting}

\subsection{Falsche Ziffer}
Falls der Parameter der Ziffer keine Ziffer ist, gibt das Programm einen Fehler aus:
\begin{lstlisting}
$ python main.py 1 -1
Error: -1 is not a digit, exiting...
\end{lstlisting}


\subsection{Zusätzliche Ausgabe}
\begin{lstlisting}
$ python main.py 2019 1 -v
looking for normal shortest
generated split table with digits: 1
generated split table with digits: 2
generated split table with digits: 3
generated split table with digits: 4
generated split table with digits: 5
generated split table with digits: 6
generated split table with digits: 7
generated split table with digits: 8
found (((1111*(1+1))-1)-((111+(1-11))*(1+1))) with 15 digits, looking if shorter is possible
generated split table with digits: 9
found (((111*(1+1))*(11-(1+1)))-((1-11)-11)) with 14 digits, looking if shorter is possible
generated split table with digits: 10
found ((((11111-1)/11)*(1+1))-1) with 11 digits, looking if shorter is possible

looking for extended shortest
generated split table with digits: 1
generated split table with digits: 2
generated split table with digits: 3
generated split table with digits: 4
generated split table with digits: 5
generated split table with digits: 6
found ((((1+1)^11)+1)+(((1+1)+1)*(1-11))) with 11 digits, looking if shorter is possible
generated split table with digits: 7
found ((((1+1)^11)+1)-((11-1)*((1+1)+1))) with 11 digits, looking if shorter is possible
generated split table with digits: 8
found (((1+1)+1)*(((111+1)*(((1+1)+1)!))+1)) with 11 digits, looking if shorter is possible
generated split table with digits: 9
found (((1+1)^11)-(((11-1)*((1+1)+1))-1)) with 11 digits, looking if shorter is possible

normal result ((((11111-1)/11)*(1+1))-1)
digits: 11

extended result (((1+1)^11)-(((11-1)*((1+1)+1))-1))
digits: 11
\end{lstlisting}

\section{Quellcode (ausschnittsweise)}
Hier drucke ich die Funktionen \texttt{add\_to\_table}, \texttt{generate}, \texttt{scan} und \texttt{find\_shortest} ab.
Die Klassendefinitionen für \texttt{Term}, \texttt{Number}, \texttt{UnaryOperation} sowie \texttt{BinaryOperation} werden ausgelassen.
Ausgelassen wird auch das Einlesen der Parameter und die Ausgabe des durch \texttt{find\_shortest} gefundenen Ergebnisses.

\lstinputlisting[language=Python]{code.py}

\end{document}
