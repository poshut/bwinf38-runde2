\subsection{Ziffer 1}
\subsubsection{2019}
Der von meinem Programm gefundene Term hat genauso viele Ziffern wie das Beispiel aus der Aufgabenstellung:
\begin{lstlisting}
$ command time -f 'Zeit: %E' python main.py 2019 1
normal result ((((11111-1)/11)*(1+1))-1)
digits: 11

extended result ((((1+1)+1)*(1-11))+(((1+1)^11)+1))
digits: 11
Zeit: 0:05.14
\end{lstlisting}
\subsubsection{2030}
\begin{lstlisting}
$ command time -f 'Zeit: %E' python main.py 2030 1
normal result ((1-11)*(((1+1)*((11-1)-111))-1))
digits: 12

extended result ((((1+1)^11)-((((1+1)+1)!)+1))-11)
digits: 10
Zeit: 0:01.69
\end{lstlisting}
\subsubsection{2080}
\begin{lstlisting}
$ command time -f 'Zeit: %E' python main.py 2080 1
normal result ((1-11)*((11+1)+((1+1)*(1-111))))
digits: 12

extended result (((1+1)^11)+((1+1)^((((1+1)+1)!)-1)))
digits: 10
Zeit: 0:01.42
\end{lstlisting}
\subsubsection{2980}
\begin{lstlisting}
$ command time -f 'Zeit: %E' python main.py 2980 1
normal result ((1-11)*((1+1)+(((1+1)+1)*(11-111))))
digits: 13

extended result ((((((1+1)+1)!)+1)!)-(((1+1)^11)+(11+1)))
digits: 11
Zeit: 0:05.88
\end{lstlisting}
\subsection{Ziffer 2}
\subsubsection{2019}
Der von meinem Programm gefundene Term hat genauso viele Ziffern wie das Beispiel aus der Aufgabenstellung:
\begin{lstlisting}
$ command time -f 'Zeit: %E' python main.py 2019 2
normal result ((((((22*2)+2)*22)-2)*2)-(2/2))
digits: 10

extended result (((((2^2)*2)!)/(22-2))+(2+(2/2)))
digits: 9
Zeit: 0:02.46
\end{lstlisting}
\subsubsection{2030}
\begin{lstlisting}
$ command time -f 'Zeit: %E' python main.py 2030 2
normal result (2+(((((22*2)+2)*22)+2)*2))
digits: 9

extended result ((2^2)-(22-(2^(22/2))))
digits: 8
Zeit: 0:00.53
\end{lstlisting}
\subsubsection{2080}
\begin{lstlisting}
$ command time -f 'Zeit: %E' python main.py 2080 2
normal result ((2*2)*((22+(2*2))*(22-2)))
digits: 9

extended result ((((((2^2)!)*2)^2)-2)-222)
digits: 8
Zeit: 0:00.38
\end{lstlisting}
\subsubsection{2980}
\begin{lstlisting}
$ command time -f 'Zeit: %E' python main.py 2980 2
normal result (2*(2-((22+2)*((2*(2-22))-22))))
digits: 11

extended result (((((2^2)!)*2)^2)+((((2^2)!)+2)^2))
digits: 8
Zeit: 0:03.07
\end{lstlisting}
\subsection{Ziffer 3}
\subsubsection{2019}
Der von meinem Programm gefundene Term hat genauso viele Ziffern wie das Beispiel aus der Aufgabenstellung:
\begin{lstlisting}
$ command time -f 'Zeit: %E' python main.py 2019 3
normal result (((333+3)*(3+3))+3)
digits: 7

extended result ((((3!)^3)*(3!))+(((3!)!)+3))
digits: 5
Zeit: 0:00.06
\end{lstlisting}
\subsubsection{2030}
\begin{lstlisting}
$ command time -f 'Zeit: %E' python main.py 2030 3
normal result ((333*(3+3))-((3/3)-33))
digits: 9

extended result (((((3!)!)-3)*3)-((((3!)!)+(3!))/(3!)))
digits: 6
Zeit: 0:00.21
\end{lstlisting}
\subsubsection{2080}
\begin{lstlisting}
$ command time -f 'Zeit: %E' python main.py 2080 3
normal result (((33*3)*(((3+3)*3)+3))+(3/3))
digits: 9

extended result ((((3!)!)*3)-((((3!)!)/3)/3))
digits: 5
Zeit: 0:00.12
\end{lstlisting}
\subsubsection{2980}
\begin{lstlisting}
$ command time -f 'Zeit: %E' python main.py 2980 3
normal result ((3/3)-(3*((3+3)-(333*3))))
digits: 9

extended result ((3/3)+(((((3!)!)+33)*3)+((3!)!)))
digits: 7
Zeit: 0:01.53
\end{lstlisting}
\subsection{Ziffer 4}
\subsubsection{2019}
Der von meinem Programm gefundene Term hat genauso viele Ziffern wie das Beispiel aus der Aufgabenstellung:
\begin{lstlisting}
$ command time -f 'Zeit: %E' python main.py 2019 4
normal result ((4-((4+4)*(4-((4*4)*(4*4)))))-(4/4))
digits: 10

extended result (((((4!)*(4!))*((4!)+4))+(4!))/(4+4))
digits: 7
Zeit: 0:02.09
\end{lstlisting}
\subsubsection{2030}
\begin{lstlisting}
$ command time -f 'Zeit: %E' python main.py 2030 4
normal result (((444+((4*4)*4))*4)-((4+4)/4))
digits: 10

extended result (((4!)/4)-((4!)-((4^4)*(4+4))))
digits: 7
Zeit: 0:02.12
\end{lstlisting}
\subsubsection{2080}
\begin{lstlisting}
$ command time -f 'Zeit: %E' python main.py 2080 4
normal result ((((4*4)*(4*4))+4)*(4+4))
digits: 7

extended result ((4+4)*((4^4)+4))
digits: 5
Zeit: 0:00.05
\end{lstlisting}
\subsubsection{2980}
\begin{lstlisting}
$ command time -f 'Zeit: %E' python main.py 2980 4
normal result ((((((4*4)*4)+4)*44)+4)-(4*4))
digits: 9

extended result ((((((4!)/4)!)+(4!))*4)+4)
digits: 5
Zeit: 0:00.24
\end{lstlisting}
\subsection{Ziffer 5}
\subsubsection{2019}
Der von meinem Programm gefundene Term hat genauso viele Ziffern wie das Beispiel aus der Aufgabenstellung:
\begin{lstlisting}
$ command time -f 'Zeit: %E' python main.py 2019 5
normal result (((((55+(5*5))*5)+5)*5)-(5+(5/5)))
digits: 10

extended result (((5!)/5)+((((5!)+5)*((5+5)+5))+(5!)))
digits: 8
Zeit: 0:05.29
\end{lstlisting}
\subsubsection{2030}
\begin{lstlisting}
$ command time -f 'Zeit: %E' python main.py 2030 5
normal result (5+((((55+(5*5))*5)+5)*5))
digits: 8

extended result (((5!)+(5*5))*(((5!)/5)-(5+5)))
digits: 7
Zeit: 0:00.52
\end{lstlisting}
\subsubsection{2080}
\begin{lstlisting}
$ command time -f 'Zeit: %E' python main.py 2080 5
normal result ((((5*5)*5)+5)*((55/5)+5))
digits: 8

extended result ((5^5)-(55*(((5!)/5)-5)))
digits: 7
Zeit: 0:00.49
\end{lstlisting}
\subsubsection{2980}
\begin{lstlisting}
$ command time -f 'Zeit: %E' python main.py 2980 5
normal result ((55*55)-(55-(5+5)))
digits: 8

extended result (((5^5)-(5!))-(5*5))
digits: 5
Zeit: 0:00.09
\end{lstlisting}
\subsection{Ziffer 6}
\subsubsection{2019}
Der von meinem Programm gefundene Term hat genauso viele Ziffern wie das Beispiel aus der Aufgabenstellung:
\begin{lstlisting}
$ command time -f 'Zeit: %E' python main.py 2019 6
normal result (((((666+6)*6)+6)*6)/(6+6))
digits: 9

extended result (((6!)+(6!))-(((((6!)+6)+((6!)/6))/6)-(6!)))
digits: 8
Zeit: 0:03.20
\end{lstlisting}
\subsubsection{2030}
\begin{lstlisting}
$ command time -f 'Zeit: %E' python main.py 2030 6
normal result (((66*6)+((66-6)/6))*(6-(6/6)))
digits: 10

extended result ((((6^6)+(6!))/(6*6))-(6-(6!)))
digits: 7
Zeit: 0:00.75
\end{lstlisting}
\subsubsection{2080}
\begin{lstlisting}
$ command time -f 'Zeit: %E' python main.py 2080 6
normal result (((6-(6*6))-((6+6)/6))*((6/6)-66))
digits: 10

extended result (((6!)+(6!))+((6!)-(((6!)+(6!))/((6+6)+6))))
digits: 8
Zeit: 0:03.53
\end{lstlisting}
\subsubsection{2980}
\begin{lstlisting}
$ command time -f 'Zeit: %E' python main.py 2980 6
normal result (((((6*6)*(6+6))-6)*(6+(6/6)))-((6+6)/6))
digits: 11

extended result (((((6!)-((6!)/6))/6)-((6!)+(6!)))+((6!)*6))
digits: 8
Zeit: 0:05.08
\end{lstlisting}
\subsection{Ziffer 7}
\subsubsection{2019}
Der von meinem Programm gefundene Term hat genauso viele Ziffern wie das Beispiel aus der Aufgabenstellung:
\begin{lstlisting}
$ command time -f 'Zeit: %E' python main.py 2019 7
normal result (((77-7)/7)-(7*(7+(7*(7-(7*7))))))
digits: 10

extended result (((((7!)+7)*(7+7))+7)/((7*7)-(7+7)))
digits: 9
Zeit: 0:32.72
\end{lstlisting}
\subsubsection{2030}
\begin{lstlisting}
$ command time -f 'Zeit: %E' python main.py 2030 7
normal result (((7-(7*(7-(7*7))))*7)-77)
digits: 8

extended result ((((7!)/(7+7))+(7-77))*7)
digits: 7
Zeit: 0:02.59
\end{lstlisting}
\subsubsection{2080}
\begin{lstlisting}
$ command time -f 'Zeit: %E' python main.py 2080 7
normal result ((7-(77*(7-((7+7)*(7+7)))))/7)
digits: 9

extended result ((((((7*7)+7)*7)+7)*7)-(((7!)/7)-7))
digits: 9
Zeit: 0:31.40
\end{lstlisting}
\subsubsection{2980}
\begin{lstlisting}
$ command time -f 'Zeit: %E' python main.py 2980 7
normal result ((((7/7)-7)*((7*(7-77))-7))-((7+7)/7))
digits: 11

extended result (((7!)+((7*7)*(7-(7*7))))-((7+7)/7))
digits: 9
Zeit: 0:36.93
\end{lstlisting}
\subsection{Ziffer 8}
\subsubsection{2019}
Der von meinem Programm gefundene Term hat genauso viele Ziffern wie das Beispiel aus der Aufgabenstellung:
\begin{lstlisting}
$ command time -f 'Zeit: %E' python main.py 2019 8
normal result ((((8-(8*(8-((8*8)*8))))*8)-(8+8))/(8+8))
digits: 11

extended result ((((8!)/(8+8))-((8*8)*8))+(88/8))
digits: 9
Zeit: 0:35.46
\end{lstlisting}
\subsubsection{2030}
\begin{lstlisting}
$ command time -f 'Zeit: %E' python main.py 2030 8
normal result (((8+8)/8)*(((8*8)*(8+8))-(8+(8/8))))
digits: 10

extended result ((88*(8+8))-(8-(((8!)/8)/8)))
digits: 8
Zeit: 0:03.04
\end{lstlisting}
\subsubsection{2080}
\begin{lstlisting}
$ command time -f 'Zeit: %E' python main.py 2080 8
normal result ((((8+8)*8)+((8+8)/8))*(8+8))
digits: 8

extended result (((8!)/(((88+8)/8)+8))+(8*8))
digits: 8
Zeit: 0:02.73
\end{lstlisting}
\subsubsection{2980}
\begin{lstlisting}
$ command time -f 'Zeit: %E' python main.py 2980 8
normal result (((((8*8)*8)*((8*8)*8))+(88+8))/88)
digits: 11

extended result ((((8!)+(888*8))/(8+8))+(8+8))
digits: 9
Zeit: 0:36.19
\end{lstlisting}
\subsection{Ziffer 9}
\subsubsection{2019}
Der von meinem Programm gefundene Term hat genauso viele Ziffern wie das Beispiel aus der Aufgabenstellung:
\begin{lstlisting}
$ command time -f 'Zeit: %E' python main.py 2019 9
normal result ((((999+9)*(9+9))+((9+9)+9))/9)
digits: 10

extended result (99+((9!)/((99+9)+(9*9))))
digits: 8
Zeit: 0:04.59
\end{lstlisting}
\subsubsection{2030}
\begin{lstlisting}
$ command time -f 'Zeit: %E' python main.py 2030 9
normal result (((9*9)-(99/9))*((9+9)+(99/9)))
digits: 10

extended result ((((9+9)/9)^(99/9))-(9+9))
digits: 8
Zeit: 0:03.10
\end{lstlisting}
\subsubsection{2080}
\begin{lstlisting}
$ command time -f 'Zeit: %E' python main.py 2080 9
normal result ((9/9)+(99*(((99+9)/9)+9)))
digits: 9

extended result (((9*9)*(9*9))-((((9!)/9)+9)/9))
digits: 8
Zeit: 0:02.22
\end{lstlisting}
\subsubsection{2980}
\begin{lstlisting}
$ command time -f 'Zeit: %E' python main.py 2980 9
normal result (((((9+9)*9)+9)*(9+9))-(99-(9/9)))
digits: 10

extended result (((9*9)*9)+((((9!)/(9+9))+99)/9))
digits: 9
Zeit: 0:25.42
\end{lstlisting}
