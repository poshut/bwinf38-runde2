\documentclass[a4paper,10pt,ngerman]{scrartcl}
\usepackage{babel}
% \usepackage[T1]{fontenc}
\usepackage[utf8]{inputenc}
\usepackage[a4paper,margin=2.5cm]{geometry}
\usepackage{todonotes}
\usepackage{csquotes}
\usepackage[backend=biber]{biblatex}
\usetikzlibrary{positioning}
\addbibresource{sources.bib}

% Die nächsten drei Felder bitte anpassen:
\newcommand{\Name}{Richard Wohlbold} % Teamname oder eigenen Namen angeben
\newcommand{\Einsendenummer}{00487}
\newcommand{\Aufgabe}{Aufgabe 3: Abbiegen}

% Kopf- und Fußzeilen
\usepackage{scrlayer-scrpage}
\setkomafont{pageheadfoot}{\textrm}
\ifoot{\Name}
\cfoot{\thepage}
\chead{\Aufgabe}
\ofoot{Team-ID: \Einsendenummer}

% Für mathematische Befehle und Symbole
\usepackage{amsmath}
\usepackage{amssymb}

% Für Bilder
\usepackage{graphicx}

% Für Algorithmen
\usepackage{algpseudocode}
\usepackage{algorithm}

% Für Quelltext
\usepackage{listings}
\usepackage{color}
\definecolor{mygreen}{rgb}{0,0.6,0}
\definecolor{mygray}{rgb}{0.5,0.5,0.5}
\definecolor{mymauve}{rgb}{0.58,0,0.82}
\lstset{
  keywordstyle=\color{blue},commentstyle=\color{mygreen},
  stringstyle=\color{mymauve},rulecolor=\color{black},
  basicstyle=\footnotesize\ttfamily,numberstyle=\tiny\color{mygray},
  captionpos=b, % sets the caption-position to bottom
  keepspaces=true, % keeps spaces in text
  numbers=left, numbersep=5pt, showspaces=false,showstringspaces=true,
  showtabs=false, stepnumber=2, tabsize=2, title=\lstname,
  breaklines=true,
  postbreak=\mbox{\textcolor{red}{$\hookrightarrow$}\space},
}
\lstdefinelanguage{JavaScript}{ % JavaScript ist als einzige Sprache noch nicht vordefiniert
  keywords={break, case, catch, continue, debugger, default, delete, do, else, finally, for, function, if, in, instanceof, new, return, switch, this, throw, try, typeof, var, void, while, with},
  morecomment=[l]{//},
  morecomment=[s]{/*}{*/},
  morestring=[b]',
  morestring=[b]",
  sensitive=true
}

\lstset{literate=%
    {_}{{\_}}1
    {Ö}{{\"O}}1
    {Ä}{{\"A}}1
    {Ü}{{\"U}}1
    {ß}{{\ss}}1
    {ü}{{\"u}}1
    {ä}{{\"a}}1
    {ö}{{\"o}}1
    {~}{{\textasciitilde}}1
    {“}{{"}}1
    {„}{{"}}1
}

% Diese beiden Pakete müssen als letztes geladen werden
%\usepackage{hyperref} % Anklickbare Links im Dokument
\usepackage{cleveref}

% Daten für die Titelseite
\title{\Aufgabe}
\author{\Name\\Team-ID: \Einsendenummer}
\date{\today}



\begin{document}

\maketitle
\tableofcontents

\section{Lösungsidee}
\label{sec:idea}
Für die Beantwortung von Bilals Frage muss das Straßennetz zunächst als Datenstruktur repräsentiert werden.
Daraufhin arbeite ich mit einer modifizierten Version des Dijkstra-Shortest-Path-Algorithmus, um einen Weg zu finden, dessen Länge innerhalb der gegebenen Grenzen liegt und auf dem so wenig wie möglich abgebogen werden muss.

\subsection{Darstellung des Straßennetzes}
\label{sec:idea:representation}
Straßennetze werden allgemein als gewichtete Graphen, d.h. Mengen von Knoten und Kanten zwischen diesen, repräsentiert.
Normalerweise entsprechen dabei Knoten den Kreuzungen und Kanten den Straßen zwischen den Kreuzungen, ein sog. \textit{node-based-graph}.
Dabei hat jede Kante ein Gewicht, d.h. einen ihr zugewiesenen numerischen Wert, der meistens der Länge der entsprechenden Straße entspricht.
Das Beispiel aus der Aufgabenstellung ist in Abbildung \ref{fig:graph-idee} als \textit{node-based-graph} repräsentiert.

\todo{weights for the edges}
\begin{figure}
\centering
\begin{tikzpicture}
    [auto=left,every node/.style={circle,fill=blue!20}]
    \node (0) at (0,0) {0};
    \node (1) at (1,0)  {1};
    \node (2) at (2,0) {2};
    \node (3) at (3,0) {3};
    \node [style={fill=green!60}] (4) at (4,0) {4};
    \node (5) at (0,-1) {5};
    \node (6) at (2,-1) {6};
    \node (7) at (0,-2) {7};
    \node (8) at (1,-2) {8};
    \node [style={fill=red!60}] (9) at (0,-3) {9}; 
    \draw (0) -- (1);
    \draw (1) -- (2);
    \draw (2) -- (3);
    \draw (3) -- (4);
    \draw (5) -- (0);
    \draw (5) -- (7);
    \draw (5) -- (1);
    \draw (7) -- (9);
    \draw (7) -- (8);
    \draw (8) -- (6);
    \draw (6) -- (1);
    \draw (6) -- (2);
    \draw (6) -- (3);
\end{tikzpicture}
\label{fig:graph-idee}
\caption{Beispiel aus der Aufgabe, als \textit{node-based-graph}, zur Übersichtlichkeit ohne entsprechende Gewichte, repräsentiert. Knoten 9 ist dabei der Ausgangsknoten und Knoten 4 der Zielknoten}
\end{figure}


Wenn jedoch auch das Abbiegen miteinbezogen werden soll, wird die Darstellung des Straßennetzes schwieriger.
Oft soll beispielsweise auf einer Route weniger abgebogen werden, weshalb Abbiegevorgängen Gewichte zugeordnet werden, sogenannte \textit{turn costs} \cite{geisberger}.
Bei diesen Gewichten stößt die traditionelle Repräsentation der Straßennetze an ihre Grenzen, sodass bei \textit{turn costs} der Ansatz eines \textit{edge-based-graph}s gewählt wird.
Dabei wird jede Straße als Knoten repräsentiert.
Für jede Möglichkeit, von einer Straße auf eine andere zu gelangen, wird eine Kante zwischen den zwei entsprechenden Knoten hinzugefügt.
Falls zwischen den Straßen abgebogen werden muss, kann dies entweder als Gewicht der Kante dargestellt werden oder ein einfaches \textsc{Wahr} oder \textsc{Falsch} für jede Kante gespeichert werden.
Die Länge jeder Straße muss für einen \textit{edge-based-graph} für jeden Knoten statt für jede Kante gespeichert werden.
Das Beispiel aus der Aufgabe ist in Abbildung \ref{fig:edge-based-graph} als \textit{edge-based-graph} repräsentiert.

\begin{figure}
\centering
\begin{tikzpicture}
    [auto=left,every node/.style={circle,fill=blue!20}]
    \node [style={fill=green!60}] (3-4) at (0,0) {3-4};
    \node (2-3) [left=of 3-4] {2-3};
    \node (1-2) [left=of 2-3] {1-2};
    \node (0-1) [left=of 1-2] {0-1};
    \node (2-6) [below=of 2-3] {2-6};
    \node (1-6) [below=of 2-6] {1-6};
    \node (1-5) [left=of 1-6] {1-5};
    \node (0-5) [left=of 1-5] {0-5};
    \node (3-6) [right=of 2-6] {3-6};
    \node (5-7) [below=of 1-5] {5-7};
    \node (7-8) [below=of 1-6] {7-8};
    \node (6-8) [right=of 1-6] {6-8};
    \node [style={fill=red!60}] (7-9) [below=of 5-7] {7-9};
    \draw (0-1) -- (1-2); % yes
    \draw (0-1) -- (1-6); % no
    \draw (0-1) -- (1-5); % no
    \draw (0-1) -- (0-5); % no
    \draw (0-5) -- (5-7); % yes
    \draw (0-5) -- (1-5); % no
    \draw (1-2) -- (1-6); % no
    \draw (1-2) -- (1-5); % no
    \draw (1-2) -- (2-6); % no
    \draw (1-2) -- (2-3); % yes
    \draw (1-5) -- (5-7); % no
    \draw (1-5) -- (1-6); % no
    \draw (1-6) -- (2-6); % no
    \draw (1-6) -- (3-6); % no
    \draw (1-6) -- (6-8); % no
    \draw (2-3) -- (2-6); % no
    \draw (2-3) -- (3-4); % yes
    \draw (2-3) -- (3-6); % no
    \draw (2-6) -- (3-6); % no
    \draw (2-6) -- (6-8); % no
    \draw (3-6) -- (6-8); % no
    \draw (5-7) -- (7-8); % no
    \draw (5-7) -- (7-9); % yes
    \draw (6-8) -- (7-8); % no
    \draw (7-8) -- (7-9); % no

\end{tikzpicture}
\caption{Beispiel aus der Aufgabe, als \textit{edge-based-graph} repräsentiert, hier ohne Gewicht}
\label{fig:edge-based-graph}
\end{figure}

\subsection{Finden des optimalen Weges}
Das Finden des optimalen Weges löse ich mithilfe des Dijkstra-Shortest-Path-Algorithmus.
Dieser benutzt eine \textsc{PriorityQueue}, um Kreuzungen geordnet nach ihrer Distanz zum Ausgangspunkt zu \enquote{besuchen}.
Wird eine Kreuzung "besucht", werden alle Kreuzungen, die von dieser Kreuzung über eine Straße erreicht werden können, der \textsc{PriorityQueue} mit ihrer Distanz vom Ursprung hinzugefügt.
Wird die Zielkreuzung gefunden, terminiert der Algorithmus und der kürzeste Weg kann rekonstruiert werden.

Der erste Schritt meines Programms ist es, den Dijkstra-Shortest-Path-Algorithmus auf den in Abschnitt \ref{sec:idea:representation} beschriebenen \textit{edge-based-graph} anzuwenden.
Dadurch wird die kürzeste Distanz vom Start zum Ziel ermittelt.
Zusätzlich kann jeweils die Anzahl an Abbiegevorgängen neben der Distanz in die \textsc{PriorityQueue} hinzugefügt werden, damit auch die maximale Anzahl an Abbiegevorgängen $n$ ermittelt werden kann.

Im Folgenden soll nun versucht werden, den kürzesten Weg mit einer gewissen Anzahl an Abbiegevorgängen $m$ zu finden.
Anfangs ist $m = n - 1$, falls ein solcher Weg existiert und Bilals Toleranz nicht überschreitet, wird $m$ um jeweils 1 verringert, bis eine der Bedingungen verletzt ist.
Dadurch wird der Weg mit der niedrigsten Anzahl an Abbiegevorgängen, der Bilals Anforderungen genügt, gefunden.

Wie auch anfangs, verfolgt die modifizierte Version des Dijkstra-Algorithmus die Anzahl der Abbiegevorgänge, die benötigt werden, um auf eine Straße (zu einem Knoten) zu gelangen.
Anders als zuvor fügt jedoch der Algorithmus der \textsc{PriorityQueue} keine Straßen hinzu, für die $>m$ Mal abgebogen werden muss.
Zusätzlich wird es erlaubt, Knoten öfter zu besuchen, sofern bei den neuen Besuchen die Zahl der Abbiegevorgänge auf der Route geringer ist, als vorher.

Somit werden nur Routen gefunden, die maximal $m$ Abbiegevorgänge haben.
Auch werden keine Routen ignoriert, die zwar etwas länger sind, aber trotzdem die entsprechende Anzahl an Abbiegevorgängen aufweisen könnten.

\section{Umsetzung}

Ich habe die Lösungsidee in Python 3 umgesetzt.

\subsection{Aus der Eingabedatei zum \textit{node-based-graph}}

Dazu wird zuerst die Eingabedatei in einen \textit{node-based-graph} eingelesen.
Die eingelesenen Kreuzungen werden als \texttt{dict} repräsentiert: Dabei erhält jede Kreuzung einen Schlüssel, die sie eindeutig identifiziert.
Die Schlüssel der Kreuzung sind jeweils die Schlüssel des \texttt{dict}s.
Die Werte des \texttt{dict}s sind die Koordinaten der Kreuzung.

Auch die Straßen werden als \texttt{dict} repräsentiert: Dabei werden als Schüssel der Schlüssel der Startkreuzung der Straße verwendet und als Wert eine Menge mit den Schlüsseln aller Kreuzungen, mit denen die Startkreuzung verbunden ist.

Das Beispiel aus der Aufgabe wird folgendermaßen repräsentiert:

\begin{lstlisting}[language=Python]
junctions = {
  0: (0, 0), 
  1: (0, 1), 
  2: (0, 2), 
  3: (0, 3), 
  4: (1, 1), 
  5: (1, 3), 
  6: (2, 2), 
  7: (2, 3), 
  8: (3, 3), 
  9: (4, 3)
} 
 
roads = {
  0: {1}, 
  1: {0, 2, 4}, 
  2: {1, 3, 4, 5}, 
  3: {2, 5}, 
  4: {1, 2, 6}, 
  5: {2, 3, 6, 7}, 
  6: {8, 4, 5, 7}, 
  7: {8, 5, 6}, 
  8: {9, 6, 7}, 
  9: {8}
}
\end{lstlisting}

\subsection{Vom node-based-graph zum edge-based-graph}
Als nächster Schritt wandelt das Programm den \textit{node-based-graph} in einen \textit{edge-based-graph} um.
Alle Straßen erhalten zuerst einheitliche Namen: Dazu werden die Schlüssel ihrer Start- und Endkreuzungen sortiert und mit einem Unterstrich verbunden.
Daraufhin werden sie mit ihrer Länge als Knoten in einem \texttt{dict} (\texttt{nodes}) gespeichert.
Die Kanten des \textit{edge-based-graphs} bilden die Kreuzungen.
Dabei wird jede Möglichkeit, auf eine andere Straße zu wechseln, einzeln zusammen mit der Information, ob für diesen Wechsel abgebogen werden muss, gespeichert.
Dies geschieht erneut in einem \texttt{dict}, dessen Schlüssel der Name einer Straße bildet.
Der Wert ist jeweils eine Liste von \texttt{tuple}n, in denen der Name der nächsten Straße und ob man geradeaus weiterfahren muss, gespeichert ist.
Durch die Umwandlung in einen \textit{edge-based-graph} gibt es nicht mehr nurnoch ein Ziel, sondern eine Menge an Straßen, von denen man das Ziel sofort erreicht.
Im Programm wird dies als \textit{set} repräsentiert, das die Namen aller Straßen, die mit der Zielkreuzung verbunden sind, enthält.
Analog wird auch ein \texttt{set} erstellt, das die Namen aller Ausgangsstraßen enthält.

Um herauszufinden, ob man bei einem Straßenwechsel abbiegen muss, wird der Winkel zwischen den beiden Vektoren der Straßen berechnet.
Es gilt:

\begin{align*}
  \vec{a}*\vec{b} &= |\vec{a}||\vec{b}|\cos{\alpha} \\
  \Leftrightarrow \cos{\alpha} &= \frac{\vec{a}*\vec{b}}{|\vec{a}||\vec{b}|}
\end{align*}

Da $\alpha = 180^{\circ}$
\begin{align*}
  \frac{\vec{a}*\vec{b}}{|\vec{a}||\vec{b}|} &= -1 \\
\end{align*}


Für das Beispiel aus der Aufgabe sehen die Datenstrukturen folgendermaßen aus (mit Auslassungen):

\begin{lstlisting}[language=Python]
nodes = {
  '0_1': 1.0, 
  '1_2': 1.0, 
  '1_4': 1.0,
  '2_3': 1.0, 
  '2_4': 1.4142135623730951,
  '2_5': 1.4142135623730951, 
  '3_5': 1.0,
  '4_6': 1.4142135623730951, 
  '5_6': 1.4142135623730951, 
  '5_7': 1.0, 
  '6_8': 1.4142135623730951,
  '6_7': 1.0, 
  '7_8': 1.0, 
  '8_9': 1.0
 } 

edges = {
 '0_1': {
  ('1_2', True), 
  ('1_4', False)
}, 
'1_2': {
  ('2_5', False), 
  ('2_3', True), 
  ('2_4', False), 
  ('0_1', True), 
  ('1_4', False)
  }, 
'1_4': {
  ('2_4', False), 
  ('4_6', False), 
  ('0_1', False), 
  ('1_2', False)}, 
'2_3': {
  ('2_4', False), 
  ('1_2', True), 
  ('3_5', False), 
  ('2_5', False)
}, 
...
} 

sources = {'0_1'}

targets = {'8_9'}
\end{lstlisting}

%  '2_4': {
%    ('2_5', False), 
%    ('1_2', False), 
%    ('2_3', False), 
%    ('4_6', False), 
%    ('1_4', False)}, 
%  '2_5': {
%    ('1_2', False), 
%    ('2_4', False), 
%    ('5_7', False), 
%    ('5_6', False), 
%    ('2_3', False), 
%    ('3_5', False)
%  }, 
%  '3_5': {
%    ('2_5', False), 
%    ('5_7', True), 
%    ('5_6', False), 
%    ('2_3', False)
%  }, 
% '4_6': {
%   ('6_8', True), 
%   ('2_4', False), 
%   ('5_6', False), 
%   ('6_7', False), 
%   ('1_4', False)
%   }, 
% '5_6': {
%   ('2_5', False), 
%   ('5_7', False), 
%   ('6_8', False), 
%   ('4_6', False), 
%   ('6_7', False), 
%   ('3_5', False)
% }, 
% '5_7': {
%   ('2_5', False), 
%   ('5_6', False), 
%   ('3_5', True), 
%   ('6_7', False), 
%   ('7_8', True)
% }, 
% '6_8': {
%   ('4_6', True), 
%   ('5_6', False), 
%   ('7_8', False), 
%   ('6_7', False), 
%   ('8_9', False)
% }, 
% '6_7': {
%   ('5_7', False), 
%   ('6_8', False), 
%   ('5_6', False), 
%   ('7_8', False), 
%   ('4_6', False)
% }, 
% '7_8': {
%   ('6_7', False), 
%   ('5_7', True), 
%   ('6_8', False), 
%   ('8_9', True)
% }, 
% '8_9': {
%   ('6_8', False), 
%   ('7_8', True)
%   }

Im Folgenden wird der Algorithmus aus Abschnitt \ref{sec:idea} ausgeführt.
Dazu wird eine Funktion \texttt{dijkstra} definiert, der den kürzesten Weg zum Ziel mit höchstens \texttt{number\_turns} Abbiegevorgängen findet und zurückgibt.
Diese arbeitet mithilfer der \texttt{PriorityQueue} aus dem Paket \texttt{queue}.
Ein Element der PriorityQueue ist ein \texttt{tuple} und enthält die bisherige Weglänge (Priorität, mit der die \texttt{PriorityQueue} die Elemente ordnet, nach dem Dijkstra-Shortest-Path-Algorithmus), den vorherigen Knoten des \textit{edge-based-graph}s sowie der gesamte Weg bis zum Knoten und die Anzahl an Abbiegevorgängen.
Wird ein neuer Knoten von dem vorherigen aus erreicht, werden die Gesamtweglänge, der vorherige Knoten, der Gesamtweg und die Anzahl an Abbiegevorgängen entsprechend angepasst und verändert der PriorityQueue hinzugefügt.

Falls auf einem Weg bereits \texttt{number\_turns} abgebogen wurde und für den nächsten Knoten ein Abbiegevorgang benötigt wird, wird dieser nicht der \texttt{PriorityQueue} hinzugefügt, sodass nur Wege mit einer Maximalanzahl an Abbiegevorgängen von \texttt{number\_turns} zurückgegeben werden können.

Ist die \texttt{PriorityQueue} leer, so existiert kein Weg mit \texttt{number\_turns} oder weniger Abbiegevorgängen.
Die Funktion \texttt{dijkstra} gibt entsprechend \texttt{None} zurück.

Wird kein Weg mit $m$ Abbiegevorgängen gefunden oder überschreitet dessen Länge Bilals Toleranz, wird der letztgefundene Weg verwendet.
Zur Ausgabe müssen die Knoten des \textit{edge-based-graph}s wieder in Knoten des \textit{node-based-graph}s umgewandelt werden.
Dazu werden die Schlüssel der Knoten des \textit{edge-based-graph}s aufgeteilt, sodass je zwei Schlüssel des \textit{node-based-graph}s erhalten werden (siehe oben).
Nun wird überprüft, von welcher Kreuzung Bilal auf die Straße fährt und die neue Kreuzung wird an die Liste an Kreuzungsschlüsseln \texttt{path} angehängt.
Mithilfe des \texttt{dict}s \texttt{junctions} werden nun den Kreuzungsschlüsseln Koordinaten zugeordnet, die mit der Weglänge und der Anzahl an Abbiegevorgängen ausgegeben werden.

\subsection{Visualisierung}
Mithilfe des \LaTeX-Pakets \texttt{tikz} lassen sich Graphen einfach visualisieren.
Im Python-Code ist deshalb auch eine Funktion \texttt{visualize} enthalten, die eine Datei \texttt{\_visualize.tex} erstellt, die, durch \LaTeX kom\-pi\-liert, eine grafische Repräsentation der Eingabedatei zeigt.
Bei der Visualisierung wird die Startkreuzung grün markiert und die Endkreuzung rot markiert.
$x$ und $y$-Koordinaten können wie im ersten Quadranten in einem Koordinatensystem verwendet werden.

\subsection{Laufzeit}
Die Laufzeit des Programms ist in allen getesteten Beispielen kein Problem.
Selbst für das längste Beispiel benötigt das Programm weniger als $0.2 \textrm{s}$.

\section{Beispiele}
Zu jedem Beispiel drucke ich die grafische Repräsentation des Beispiels sowie die Programmausgabe und die Zeit, die zur Ausführung benötigt wird

\subsection{Beispiel aus der Aufgabenstellung}
Die Ergebnisse des Programms beim Beispiel aus der Aufgabenstellung entsprechen in allen Aspekten der Aufgabenstellung.
\subsubsection{Visualisierung}
\begin{tikzpicture}
[auto=left,every node/.style={circle,fill=blue!20}]
\node [style={fill=green!60}] (0) at (0, 0) {0};
\node (1) at (0, 1) {1};
\node (2) at (0, 2) {2};
\node (3) at (0, 3) {3};
\node (4) at (1, 1) {4};
\node (5) at (1, 3) {5};
\node (6) at (2, 2) {6};
\node (7) at (2, 3) {7};
\node (8) at (3, 3) {8};
\node [style={fill=red!60}] (9) at (4, 3) {9};
\draw (0) -- (1);
\draw (1) -- (0);
\draw (1) -- (2);
\draw (1) -- (4);
\draw (2) -- (1);
\draw (2) -- (3);
\draw (2) -- (4);
\draw (2) -- (5);
\draw (3) -- (2);
\draw (3) -- (5);
\draw (4) -- (1);
\draw (4) -- (2);
\draw (4) -- (6);
\draw (5) -- (2);
\draw (5) -- (3);
\draw (5) -- (6);
\draw (5) -- (7);
\draw (6) -- (8);
\draw (6) -- (4);
\draw (6) -- (5);
\draw (6) -- (7);
\draw (7) -- (8);
\draw (7) -- (5);
\draw (7) -- (6);
\draw (8) -- (9);
\draw (8) -- (6);
\draw (8) -- (7);
\draw (9) -- (8);
\end{tikzpicture}

\subsubsection{0\% Toleranz}
\begin{lstlisting}
$ python main.py ../../material/a3-abbiegen/abbiegen0.txt 0
Distance 5.83
3 turns
(0, 0) -> (0, 1) -> (1, 1) -> (2, 2) -> (3, 3) -> (4, 3)
\end{lstlisting}

\subsubsection{15\% Toleranz}
\begin{lstlisting}
$ python main.py ../../material/a3-abbiegen/abbiegen0.txt 15
Distance 6.41
2 turns
\end{lstlisting}

\subsubsection{30\% Toleranz}
\begin{lstlisting}
$ python main.py ../../material/a3-abbiegen/abbiegen0.txt 30
Distance 7.0
1 turn
(0, 0) -> (0, 1) -> (0, 2) -> (0, 3) -> (1, 3) -> (2, 3) -> (3, 3) -> (4, 3)
\end{lstlisting}

\subsection{Beispiel 1 von der Website}
\subsubsection{Visualisierung}
\begin{tikzpicture}
[auto=left,every node/.style={circle,fill=blue!20}]
\node (0) at (2, 0) {0};
\node (1) at (4, 1) {1};
\node (2) at (1, 3) {2};
\node (3) at (0, 3) {3};
\node (4) at (3, 4) {4};
\node (5) at (0, 1) {5};
\node (6) at (1, 2) {6};
\node (7) at (2, 1) {7};
\node (8) at (1, 0) {8};
\node (9) at (5, 2) {9};
\node (10) at (5, 3) {10};
\node (11) at (10, 2) {11};
\node (12) at (10, 3) {12};
\node (13) at (14, 2) {13};
\node (14) at (14, 3) {14};
\node (15) at (11, 3) {15};
\node (16) at (5, 1) {16};
\node (17) at (6, 2) {17};
\node (18) at (7, 4) {18};
\node (19) at (11, 2) {19};
\node (20) at (3, 0) {20};
\node (21) at (0, 2) {21};
\node (22) at (2, 2) {22};
\node (23) at (3, 2) {23};
\node (24) at (3, 1) {24};
\node (25) at (8, 1) {25};
\node (26) at (9, 3) {26};
\node (27) at (12, 1) {27};
\node (28) at (13, 3) {28};
\node [style={fill=green!60}] (29) at (0, 0) {29};
\node (30) at (4, 0) {30};
\node (31) at (4, 2) {31};
\node (32) at (5, 0) {32};
\node (33) at (7, 2) {33};
\node (34) at (7, 1) {34};
\node (35) at (6, 1) {35};
\node (36) at (6, 0) {36};
\node (37) at (8, 2) {37};
\node (38) at (7, 0) {38};
\node (39) at (8, 0) {39};
\node (40) at (9, 0) {40};
\node (41) at (10, 0) {41};
\node (42) at (11, 0) {42};
\node (43) at (9, 1) {43};
\node (44) at (12, 2) {44};
\node (45) at (9, 2) {45};
\node (46) at (12, 0) {46};
\node (47) at (13, 0) {47};
\node (48) at (10, 1) {48};
\node [style={fill=red!60}] (49) at (14, 0) {49};
\node (50) at (11, 1) {50};
\node (51) at (13, 1) {51};
\node (52) at (13, 2) {52};
\node (53) at (14, 1) {53};
\node (54) at (12, 3) {54};
\node (55) at (8, 4) {55};
\node (56) at (9, 4) {56};
\node (57) at (10, 4) {57};
\node (58) at (13, 4) {58};
\node (59) at (8, 3) {59};
\node (60) at (14, 4) {60};
\node (61) at (11, 4) {61};
\node (62) at (12, 4) {62};
\node (63) at (7, 3) {63};
\node (64) at (6, 3) {64};
\node (65) at (5, 4) {65};
\node (66) at (4, 4) {66};
\node (67) at (3, 3) {67};
\node (68) at (6, 4) {68};
\node (69) at (1, 1) {69};
\node (70) at (2, 3) {70};
\node (71) at (4, 3) {71};
\node (72) at (2, 4) {72};
\node (73) at (1, 4) {73};
\node (74) at (0, 4) {74};
\draw (0) -- (24);
\draw (0) -- (1);
\draw (0) -- (20);
\draw (0) -- (8);
\draw (1) -- (0);
\draw (1) -- (10);
\draw (1) -- (16);
\draw (1) -- (20);
\draw (1) -- (24);
\draw (1) -- (30);
\draw (1) -- (31);
\draw (2) -- (3);
\draw (2) -- (4);
\draw (2) -- (5);
\draw (2) -- (72);
\draw (2) -- (21);
\draw (3) -- (72);
\draw (3) -- (73);
\draw (3) -- (2);
\draw (3) -- (21);
\draw (4) -- (72);
\draw (4) -- (2);
\draw (4) -- (66);
\draw (4) -- (70);
\draw (5) -- (21);
\draw (5) -- (2);
\draw (5) -- (29);
\draw (5) -- (6);
\draw (6) -- (29);
\draw (6) -- (5);
\draw (6) -- (70);
\draw (7) -- (69);
\draw (7) -- (8);
\draw (7) -- (22);
\draw (7) -- (24);
\draw (7) -- (31);
\draw (8) -- (24);
\draw (8) -- (0);
\draw (8) -- (29);
\draw (8) -- (7);
\draw (9) -- (16);
\draw (9) -- (10);
\draw (10) -- (64);
\draw (10) -- (65);
\draw (10) -- (1);
\draw (10) -- (9);
\draw (10) -- (23);
\draw (10) -- (31);
\draw (11) -- (40);
\draw (11) -- (43);
\draw (11) -- (12);
\draw (11) -- (15);
\draw (11) -- (48);
\draw (11) -- (19);
\draw (11) -- (25);
\draw (11) -- (26);
\draw (11) -- (61);
\draw (12) -- (26);
\draw (12) -- (11);
\draw (12) -- (61);
\draw (13) -- (51);
\draw (13) -- (53);
\draw (13) -- (14);
\draw (14) -- (28);
\draw (14) -- (13);
\draw (14) -- (60);
\draw (14) -- (52);
\draw (15) -- (11);
\draw (15) -- (19);
\draw (15) -- (54);
\draw (15) -- (61);
\draw (15) -- (62);
\draw (16) -- (32);
\draw (16) -- (1);
\draw (16) -- (33);
\draw (16) -- (64);
\draw (16) -- (9);
\draw (16) -- (17);
\draw (16) -- (30);
\draw (17) -- (64);
\draw (17) -- (33);
\draw (17) -- (16);
\draw (17) -- (18);
\draw (17) -- (59);
\draw (18) -- (64);
\draw (18) -- (17);
\draw (18) -- (68);
\draw (18) -- (55);
\draw (19) -- (11);
\draw (19) -- (44);
\draw (19) -- (15);
\draw (19) -- (48);
\draw (19) -- (54);
\draw (19) -- (28);
\draw (20) -- (0);
\draw (20) -- (1);
\draw (21) -- (2);
\draw (21) -- (3);
\draw (21) -- (5);
\draw (22) -- (67);
\draw (22) -- (7);
\draw (22) -- (69);
\draw (22) -- (23);
\draw (23) -- (10);
\draw (23) -- (71);
\draw (23) -- (22);
\draw (23) -- (31);
\draw (24) -- (8);
\draw (24) -- (0);
\draw (24) -- (1);
\draw (24) -- (7);
\draw (25) -- (37);
\draw (25) -- (38);
\draw (25) -- (11);
\draw (25) -- (45);
\draw (25) -- (26);
\draw (26) -- (33);
\draw (26) -- (11);
\draw (26) -- (12);
\draw (26) -- (45);
\draw (26) -- (25);
\draw (26) -- (61);
\draw (26) -- (57);
\draw (27) -- (41);
\draw (27) -- (42);
\draw (27) -- (44);
\draw (27) -- (50);
\draw (27) -- (51);
\draw (27) -- (52);
\draw (27) -- (28);
\draw (28) -- (44);
\draw (28) -- (14);
\draw (28) -- (19);
\draw (28) -- (52);
\draw (28) -- (54);
\draw (28) -- (27);
\draw (28) -- (60);
\draw (29) -- (8);
\draw (29) -- (69);
\draw (29) -- (5);
\draw (29) -- (6);
\draw (30) -- (16);
\draw (30) -- (1);
\draw (30) -- (32);
\draw (31) -- (1);
\draw (31) -- (10);
\draw (31) -- (7);
\draw (31) -- (23);
\draw (32) -- (16);
\draw (32) -- (34);
\draw (32) -- (36);
\draw (32) -- (30);
\draw (33) -- (35);
\draw (33) -- (37);
\draw (33) -- (16);
\draw (33) -- (17);
\draw (33) -- (26);
\draw (33) -- (59);
\draw (34) -- (32);
\draw (34) -- (36);
\draw (34) -- (38);
\draw (35) -- (33);
\draw (35) -- (37);
\draw (36) -- (32);
\draw (36) -- (34);
\draw (37) -- (33);
\draw (37) -- (35);
\draw (37) -- (38);
\draw (37) -- (25);
\draw (38) -- (34);
\draw (38) -- (37);
\draw (38) -- (39);
\draw (38) -- (43);
\draw (38) -- (25);
\draw (39) -- (43);
\draw (39) -- (38);
\draw (40) -- (48);
\draw (40) -- (11);
\draw (41) -- (48);
\draw (41) -- (42);
\draw (41) -- (27);
\draw (41) -- (50);
\draw (42) -- (41);
\draw (42) -- (27);
\draw (42) -- (51);
\draw (43) -- (11);
\draw (43) -- (38);
\draw (43) -- (39);
\draw (44) -- (48);
\draw (44) -- (27);
\draw (44) -- (19);
\draw (44) -- (28);
\draw (45) -- (25);
\draw (45) -- (26);
\draw (46) -- (51);
\draw (46) -- (53);
\draw (46) -- (47);
\draw (47) -- (49);
\draw (47) -- (53);
\draw (47) -- (46);
\draw (48) -- (40);
\draw (48) -- (41);
\draw (48) -- (11);
\draw (48) -- (44);
\draw (48) -- (50);
\draw (48) -- (19);
\draw (49) -- (53);
\draw (49) -- (47);
\draw (50) -- (48);
\draw (50) -- (41);
\draw (50) -- (27);
\draw (51) -- (42);
\draw (51) -- (13);
\draw (51) -- (46);
\draw (51) -- (52);
\draw (51) -- (53);
\draw (51) -- (27);
\draw (52) -- (51);
\draw (52) -- (27);
\draw (52) -- (28);
\draw (52) -- (14);
\draw (53) -- (13);
\draw (53) -- (46);
\draw (53) -- (47);
\draw (53) -- (49);
\draw (53) -- (51);
\draw (54) -- (60);
\draw (54) -- (15);
\draw (54) -- (19);
\draw (54) -- (58);
\draw (54) -- (28);
\draw (55) -- (56);
\draw (55) -- (18);
\draw (55) -- (63);
\draw (56) -- (57);
\draw (56) -- (59);
\draw (56) -- (63);
\draw (56) -- (55);
\draw (57) -- (56);
\draw (57) -- (26);
\draw (58) -- (60);
\draw (58) -- (54);
\draw (59) -- (56);
\draw (59) -- (17);
\draw (59) -- (33);
\draw (59) -- (63);
\draw (60) -- (58);
\draw (60) -- (28);
\draw (60) -- (54);
\draw (60) -- (14);
\draw (61) -- (11);
\draw (61) -- (12);
\draw (61) -- (15);
\draw (61) -- (26);
\draw (61) -- (62);
\draw (62) -- (61);
\draw (62) -- (15);
\draw (63) -- (56);
\draw (63) -- (59);
\draw (63) -- (55);
\draw (64) -- (65);
\draw (64) -- (10);
\draw (64) -- (16);
\draw (64) -- (17);
\draw (64) -- (18);
\draw (65) -- (64);
\draw (65) -- (67);
\draw (65) -- (66);
\draw (65) -- (10);
\draw (66) -- (65);
\draw (66) -- (4);
\draw (66) -- (70);
\draw (67) -- (65);
\draw (67) -- (22);
\draw (67) -- (70);
\draw (67) -- (71);
\draw (68) -- (18);
\draw (69) -- (7);
\draw (69) -- (29);
\draw (69) -- (70);
\draw (69) -- (22);
\draw (70) -- (66);
\draw (70) -- (67);
\draw (70) -- (4);
\draw (70) -- (69);
\draw (70) -- (6);
\draw (71) -- (67);
\draw (71) -- (23);
\draw (72) -- (73);
\draw (72) -- (2);
\draw (72) -- (3);
\draw (72) -- (4);
\draw (73) -- (72);
\draw (73) -- (74);
\draw (73) -- (3);
\draw (74) -- (73);
\end{tikzpicture}

\subsubsection{0\% Toleranz}
\begin{lstlisting}
$ python main.py ../../material/a3-abbiegen/abbiegen1.txt 0
Distance 17.3
6 turns
(0, 0) -> (1, 1) -> (2, 1) -> (3, 1) -> (4, 1) -> (5, 1) -> (7, 2) -> (9, 3) -> (10, 2) -> (10, 1) -> (11, 1) -> (12, 1) -> (13, 1) -> (14, 1) -> (14, 0)
\end{lstlisting}

\subsubsection{15\% Toleranz}
Mit 15\% Toleranz lässt sich die Anzahl an Abbiegevorgängen von 6 auf 5 reduzieren:
\begin{lstlisting}
$ python main.py ../../material/a3-abbiegen/abbiegen1.txt 15
Distance 19.12
5 turns
(0, 0) -> (1, 1) -> (2, 1) -> (3, 1) -> (4, 1) -> (5, 1) -> (7, 2) -> (9, 3) -> (11, 4) -> (11, 3) -> (12, 3) -> (13, 3) -> (14, 3) -> (14, 2) -> (14, 1) -> (14, 0)
\end{lstlisting}

\subsubsection{30\% Toleranz}
Auch mit 30\% Toleranz findet sich kein Weg mit weniger Abbiegevorgängen:
\begin{lstlisting}
$ python main.py ../../material/a3-abbiegen/abbiegen1.txt 30
Distance 19.12
5 turns
(0, 0) -> (1, 1) -> (2, 1) -> (3, 1) -> (4, 1) -> (5, 1) -> (7, 2) -> (9, 3) -> (11, 4) -> (11, 3) -> (12, 3) -> (13, 3) -> (14, 3) -> (14, 2) -> (14, 1) -> (14, 0)
\end{lstlisting}


\subsection{Beispiel 2 von der Website}
\subsubsection{Visualisierung}
\begin{tikzpicture}
[auto=left,every node/.style={circle,fill=blue!20}]
\node (0) at (2, 0) {0};
\node (1) at (4, 1) {1};
\node (2) at (0, 2) {2};
\node (3) at (1, 4) {3};
\node (4) at (1, 3) {4};
\node (5) at (2, 3) {5};
\node (6) at (0, 1) {6};
\node (7) at (1, 2) {7};
\node (8) at (5, 2) {8};
\node (9) at (5, 3) {9};
\node (10) at (1, 7) {10};
\node (11) at (0, 5) {11};
\node (12) at (7, 2) {12};
\node (13) at (7, 3) {13};
\node (14) at (1, 5) {14};
\node (15) at (2, 7) {15};
\node (16) at (5, 1) {16};
\node (17) at (3, 0) {17};
\node (18) at (2, 2) {18};
\node (19) at (3, 2) {19};
\node (20) at (3, 1) {20};
\node (21) at (1, 0) {21};
\node (22) at (1, 6) {22};
\node (23) at (0, 4) {23};
\node (24) at (2, 5) {24};
\node (25) at (6, 1) {25};
\node [style={fill=green!60}] (26) at (0, 0) {26};
\node (27) at (4, 0) {27};
\node (28) at (9, 7) {28};
\node (29) at (9, 6) {29};
\node (30) at (7, 1) {30};
\node (31) at (8, 5) {31};
\node (32) at (5, 0) {32};
\node (33) at (2, 1) {33};
\node (34) at (6, 2) {34};
\node (35) at (9, 5) {35};
\node (36) at (8, 4) {36};
\node (37) at (4, 2) {37};
\node (38) at (9, 4) {38};
\node (39) at (6, 0) {39};
\node (40) at (8, 3) {40};
\node (41) at (7, 0) {41};
\node (42) at (8, 2) {42};
\node (43) at (8, 7) {43};
\node (44) at (6, 6) {44};
\node (45) at (8, 6) {45};
\node (46) at (7, 5) {46};
\node (47) at (7, 4) {47};
\node (48) at (7, 6) {48};
\node (49) at (8, 0) {49};
\node (50) at (9, 2) {50};
\node (51) at (8, 1) {51};
\node (52) at (3, 6) {52};
\node (53) at (3, 7) {53};
\node (54) at (2, 6) {54};
\node (55) at (9, 1) {55};
\node (56) at (6, 3) {56};
\node (57) at (5, 7) {57};
\node [style={fill=red!60}] (58) at (9, 0) {58};
\node (59) at (4, 7) {59};
\node (60) at (5, 4) {60};
\node (61) at (3, 5) {61};
\node (62) at (2, 4) {62};
\node (63) at (6, 4) {63};
\node (64) at (4, 4) {64};
\node (65) at (4, 5) {65};
\node (66) at (4, 6) {66};
\node (67) at (6, 5) {67};
\node (68) at (5, 5) {68};
\node (69) at (3, 4) {69};
\node (70) at (6, 7) {70};
\node (71) at (7, 7) {71};
\node (72) at (5, 6) {72};
\node (73) at (0, 6) {73};
\node (74) at (9, 3) {74};
\node (75) at (4, 3) {75};
\node (76) at (1, 1) {76};
\node (77) at (3, 3) {77};
\node (78) at (0, 3) {78};
\draw (0) -- (1);
\draw (0) -- (21);
\draw (0) -- (17);
\draw (1) -- (0);
\draw (1) -- (16);
\draw (1) -- (17);
\draw (1) -- (20);
\draw (1) -- (27);
\draw (2) -- (78);
\draw (2) -- (3);
\draw (2) -- (5);
\draw (2) -- (6);
\draw (3) -- (2);
\draw (3) -- (14);
\draw (3) -- (78);
\draw (3) -- (23);
\draw (3) -- (61);
\draw (3) -- (62);
\draw (4) -- (5);
\draw (4) -- (62);
\draw (5) -- (2);
\draw (5) -- (4);
\draw (5) -- (69);
\draw (5) -- (7);
\draw (5) -- (77);
\draw (5) -- (62);
\draw (6) -- (26);
\draw (6) -- (2);
\draw (6) -- (7);
\draw (7) -- (5);
\draw (7) -- (6);
\draw (7) -- (76);
\draw (7) -- (77);
\draw (7) -- (26);
\draw (8) -- (16);
\draw (8) -- (9);
\draw (8) -- (56);
\draw (8) -- (20);
\draw (9) -- (37);
\draw (9) -- (8);
\draw (9) -- (75);
\draw (9) -- (19);
\draw (9) -- (56);
\draw (9) -- (63);
\draw (10) -- (73);
\draw (10) -- (11);
\draw (10) -- (15);
\draw (11) -- (10);
\draw (11) -- (22);
\draw (11) -- (23);
\draw (12) -- (34);
\draw (12) -- (36);
\draw (12) -- (42);
\draw (12) -- (13);
\draw (12) -- (16);
\draw (12) -- (56);
\draw (12) -- (25);
\draw (12) -- (30);
\draw (13) -- (56);
\draw (13) -- (12);
\draw (13) -- (36);
\draw (14) -- (3);
\draw (14) -- (15);
\draw (14) -- (52);
\draw (14) -- (54);
\draw (14) -- (23);
\draw (14) -- (24);
\draw (15) -- (54);
\draw (15) -- (10);
\draw (15) -- (14);
\draw (15) -- (22);
\draw (16) -- (32);
\draw (16) -- (1);
\draw (16) -- (34);
\draw (16) -- (8);
\draw (16) -- (12);
\draw (16) -- (56);
\draw (16) -- (25);
\draw (16) -- (27);
\draw (17) -- (0);
\draw (17) -- (1);
\draw (18) -- (75);
\draw (18) -- (19);
\draw (18) -- (76);
\draw (19) -- (33);
\draw (19) -- (18);
\draw (19) -- (76);
\draw (19) -- (9);
\draw (20) -- (1);
\draw (20) -- (33);
\draw (20) -- (37);
\draw (20) -- (8);
\draw (20) -- (21);
\draw (21) -- (0);
\draw (21) -- (26);
\draw (21) -- (20);
\draw (22) -- (11);
\draw (22) -- (15);
\draw (22) -- (23);
\draw (23) -- (3);
\draw (23) -- (11);
\draw (23) -- (22);
\draw (23) -- (14);
\draw (24) -- (66);
\draw (24) -- (52);
\draw (24) -- (61);
\draw (24) -- (14);
\draw (25) -- (16);
\draw (25) -- (12);
\draw (25) -- (30);
\draw (26) -- (76);
\draw (26) -- (21);
\draw (26) -- (6);
\draw (26) -- (7);
\draw (27) -- (16);
\draw (27) -- (1);
\draw (27) -- (32);
\draw (28) -- (48);
\draw (28) -- (43);
\draw (28) -- (29);
\draw (29) -- (28);
\draw (29) -- (45);
\draw (29) -- (46);
\draw (29) -- (31);
\draw (30) -- (32);
\draw (30) -- (39);
\draw (30) -- (41);
\draw (30) -- (42);
\draw (30) -- (12);
\draw (30) -- (25);
\draw (31) -- (35);
\draw (31) -- (29);
\draw (31) -- (46);
\draw (31) -- (47);
\draw (32) -- (16);
\draw (32) -- (27);
\draw (32) -- (30);
\draw (32) -- (39);
\draw (33) -- (19);
\draw (33) -- (20);
\draw (33) -- (37);
\draw (34) -- (16);
\draw (34) -- (12);
\draw (35) -- (36);
\draw (35) -- (38);
\draw (35) -- (40);
\draw (35) -- (47);
\draw (35) -- (31);
\draw (36) -- (35);
\draw (36) -- (12);
\draw (36) -- (13);
\draw (36) -- (47);
\draw (36) -- (56);
\draw (37) -- (33);
\draw (37) -- (20);
\draw (37) -- (9);
\draw (38) -- (40);
\draw (38) -- (74);
\draw (38) -- (35);
\draw (39) -- (32);
\draw (39) -- (30);
\draw (40) -- (74);
\draw (40) -- (35);
\draw (40) -- (42);
\draw (40) -- (38);
\draw (41) -- (42);
\draw (41) -- (49);
\draw (41) -- (51);
\draw (41) -- (55);
\draw (41) -- (30);
\draw (42) -- (40);
\draw (42) -- (41);
\draw (42) -- (74);
\draw (42) -- (12);
\draw (42) -- (50);
\draw (42) -- (51);
\draw (42) -- (55);
\draw (42) -- (30);
\draw (43) -- (44);
\draw (43) -- (28);
\draw (43) -- (71);
\draw (44) -- (65);
\draw (44) -- (67);
\draw (44) -- (68);
\draw (44) -- (43);
\draw (44) -- (48);
\draw (45) -- (48);
\draw (45) -- (29);
\draw (45) -- (46);
\draw (46) -- (67);
\draw (46) -- (45);
\draw (46) -- (31);
\draw (46) -- (47);
\draw (46) -- (56);
\draw (46) -- (60);
\draw (46) -- (29);
\draw (46) -- (63);
\draw (47) -- (35);
\draw (47) -- (36);
\draw (47) -- (46);
\draw (47) -- (31);
\draw (48) -- (67);
\draw (48) -- (28);
\draw (48) -- (45);
\draw (48) -- (44);
\draw (49) -- (41);
\draw (49) -- (55);
\draw (50) -- (42);
\draw (50) -- (74);
\draw (50) -- (55);
\draw (51) -- (41);
\draw (51) -- (42);
\draw (51) -- (55);
\draw (52) -- (66);
\draw (52) -- (14);
\draw (52) -- (53);
\draw (52) -- (54);
\draw (52) -- (24);
\draw (52) -- (57);
\draw (52) -- (59);
\draw (53) -- (59);
\draw (53) -- (52);
\draw (53) -- (54);
\draw (54) -- (52);
\draw (54) -- (53);
\draw (54) -- (14);
\draw (54) -- (15);
\draw (55) -- (41);
\draw (55) -- (42);
\draw (55) -- (49);
\draw (55) -- (50);
\draw (55) -- (51);
\draw (55) -- (58);
\draw (56) -- (36);
\draw (56) -- (8);
\draw (56) -- (9);
\draw (56) -- (12);
\draw (56) -- (13);
\draw (56) -- (46);
\draw (56) -- (16);
\draw (56) -- (63);
\draw (57) -- (66);
\draw (57) -- (52);
\draw (58) -- (55);
\draw (59) -- (52);
\draw (59) -- (53);
\draw (60) -- (64);
\draw (60) -- (65);
\draw (60) -- (75);
\draw (60) -- (77);
\draw (60) -- (46);
\draw (60) -- (63);
\draw (61) -- (24);
\draw (61) -- (3);
\draw (61) -- (62);
\draw (62) -- (65);
\draw (62) -- (3);
\draw (62) -- (4);
\draw (62) -- (5);
\draw (62) -- (69);
\draw (62) -- (61);
\draw (63) -- (9);
\draw (63) -- (75);
\draw (63) -- (46);
\draw (63) -- (56);
\draw (63) -- (60);
\draw (64) -- (65);
\draw (64) -- (60);
\draw (64) -- (77);
\draw (65) -- (64);
\draw (65) -- (66);
\draw (65) -- (68);
\draw (65) -- (69);
\draw (65) -- (72);
\draw (65) -- (44);
\draw (65) -- (60);
\draw (65) -- (62);
\draw (66) -- (65);
\draw (66) -- (70);
\draw (66) -- (72);
\draw (66) -- (52);
\draw (66) -- (24);
\draw (66) -- (57);
\draw (67) -- (48);
\draw (67) -- (44);
\draw (67) -- (46);
\draw (67) -- (68);
\draw (68) -- (65);
\draw (68) -- (67);
\draw (68) -- (44);
\draw (69) -- (65);
\draw (69) -- (5);
\draw (69) -- (77);
\draw (69) -- (62);
\draw (70) -- (72);
\draw (70) -- (66);
\draw (71) -- (72);
\draw (71) -- (43);
\draw (72) -- (65);
\draw (72) -- (66);
\draw (72) -- (70);
\draw (72) -- (71);
\draw (73) -- (10);
\draw (74) -- (40);
\draw (74) -- (42);
\draw (74) -- (50);
\draw (74) -- (38);
\draw (75) -- (9);
\draw (75) -- (77);
\draw (75) -- (18);
\draw (75) -- (60);
\draw (75) -- (63);
\draw (76) -- (19);
\draw (76) -- (26);
\draw (76) -- (18);
\draw (76) -- (7);
\draw (77) -- (64);
\draw (77) -- (5);
\draw (77) -- (69);
\draw (77) -- (7);
\draw (77) -- (75);
\draw (77) -- (60);
\draw (78) -- (2);
\draw (78) -- (3);
\end{tikzpicture}


\subsubsection{0\% Toleranz}
\begin{lstlisting}
$ python main.py ../../material/a3-abbiegen/abbiegen2.txt 0
Distance 10.89
6 turns
(0, 0) -> (1, 0) -> (2, 0) -> (4, 1) -> (5, 1) -> (7, 2) -> (8, 2) -> (9, 1) -> (9, 0)
\end{lstlisting}

\subsubsection{15\% Toleranz}
Mit 15\% Toleranz kann die Anzahl an Abbiegevorgängen von 6 auf 5 verringert werden:
\begin{lstlisting}
$ python main.py ../../material/a3-abbiegen/abbiegen2.txt 15
Distance 11.06
5 turns
(0, 0) -> (1, 0) -> (2, 0) -> (4, 1) -> (5, 1) -> (6, 1) -> (7, 1) -> (8, 2) -> (9, 1) -> (9, 0)
\end{lstlisting}

\subsubsection{30\% Toleranz}
Mit 30\% Toleranz kann die Anzahl an Abbiegevorgängen von 5 auf 4 verringert werden:
\begin{lstlisting}
$ python main.py ../../material/a3-abbiegen/abbiegen2.txt 30
Distance 13.06
4 turns
(0, 0) -> (1, 0) -> (2, 0) -> (4, 1) -> (5, 1) -> (6, 1) -> (7, 1) -> (8, 2) -> (9, 3) -> (9, 2) -> (9, 1) -> (9, 0)
\end{lstlisting}

\subsubsection{50\% Toleranz}
Mit 50\% Toleranz kann die Anzahl an Abbiegevorgängen von 4 auf 3 verringert werden:
\begin{lstlisting}
$ python main.py ../../material/a3-abbiegen/abbiegen2.txt 50
Distance 15.94
3 turns
(0, 0) -> (1, 2) -> (3, 3) -> (5, 4) -> (7, 5) -> (8, 5) -> (9, 5) -> (9, 4) -> (9, 3) -> (9, 2) -> (9, 1) -> (9, 0)
\end{lstlisting}


\subsection{Beispiel 3 von der Website}
Bei diesem Beispiel ist der kürzeste Weg bereits der mit der geringsten Anzahl an Abbiegevorgängen.
\subsubsection{Visualisierung}
\begin{tikzpicture}
[auto=left,every node/.style={circle,fill=blue!20}]
\node (0) at (2, 0) {0};
\node (1) at (4, 1) {1};
\node (2) at (1, 3) {2};
\node (3) at (0, 3) {3};
\node (4) at (0, 1) {4};
\node (5) at (1, 2) {5};
\node (6) at (2, 1) {6};
\node (7) at (1, 0) {7};
\node (8) at (5, 2) {8};
\node (9) at (5, 3) {9};
\node (10) at (10, 2) {10};
\node (11) at (10, 3) {11};
\node (12) at (14, 2) {12};
\node (13) at (14, 3) {13};
\node (14) at (11, 3) {14};
\node (15) at (5, 1) {15};
\node (16) at (11, 2) {16};
\node (17) at (3, 0) {17};
\node (18) at (0, 2) {18};
\node (19) at (2, 2) {19};
\node (20) at (3, 2) {20};
\node (21) at (3, 1) {21};
\node (22) at (8, 1) {22};
\node (23) at (9, 3) {23};
\node (24) at (12, 1) {24};
\node (25) at (13, 3) {25};
\node [style={fill=green!60}] (26) at (0, 0) {26};
\node (27) at (4, 0) {27};
\node (28) at (4, 2) {28};
\node (29) at (5, 0) {29};
\node (30) at (6, 2) {30};
\node (31) at (7, 2) {31};
\node (32) at (7, 1) {32};
\node (33) at (6, 1) {33};
\node (34) at (6, 0) {34};
\node (35) at (8, 2) {35};
\node (36) at (7, 0) {36};
\node (37) at (8, 0) {37};
\node (38) at (9, 0) {38};
\node (39) at (10, 0) {39};
\node (40) at (11, 0) {40};
\node (41) at (9, 1) {41};
\node (42) at (12, 2) {42};
\node (43) at (9, 2) {43};
\node (44) at (12, 0) {44};
\node (45) at (13, 0) {45};
\node (46) at (10, 1) {46};
\node [style={fill=red!60}] (47) at (14, 0) {47};
\node (48) at (11, 1) {48};
\node (49) at (13, 1) {49};
\node (50) at (13, 2) {50};
\node (51) at (14, 1) {51};
\node (52) at (12, 3) {52};
\node (53) at (8, 3) {53};
\node (54) at (7, 3) {54};
\node (55) at (6, 3) {55};
\node (56) at (1, 1) {56};
\node (57) at (2, 3) {57};
\node (58) at (4, 3) {58};
\node (59) at (3, 3) {59};
\draw (0) -- (1);
\draw (0) -- (21);
\draw (0) -- (17);
\draw (0) -- (7);
\draw (1) -- (0);
\draw (1) -- (9);
\draw (1) -- (15);
\draw (1) -- (17);
\draw (1) -- (21);
\draw (1) -- (27);
\draw (1) -- (28);
\draw (2) -- (18);
\draw (2) -- (3);
\draw (2) -- (4);
\draw (3) -- (2);
\draw (3) -- (18);
\draw (4) -- (2);
\draw (4) -- (26);
\draw (4) -- (18);
\draw (4) -- (5);
\draw (5) -- (57);
\draw (5) -- (26);
\draw (5) -- (4);
\draw (6) -- (7);
\draw (6) -- (19);
\draw (6) -- (21);
\draw (6) -- (56);
\draw (6) -- (28);
\draw (7) -- (0);
\draw (7) -- (26);
\draw (7) -- (21);
\draw (7) -- (6);
\draw (8) -- (9);
\draw (8) -- (15);
\draw (9) -- (1);
\draw (9) -- (8);
\draw (9) -- (20);
\draw (9) -- (55);
\draw (9) -- (28);
\draw (10) -- (38);
\draw (10) -- (41);
\draw (10) -- (11);
\draw (10) -- (14);
\draw (10) -- (46);
\draw (10) -- (16);
\draw (10) -- (22);
\draw (10) -- (23);
\draw (11) -- (10);
\draw (11) -- (14);
\draw (11) -- (23);
\draw (12) -- (49);
\draw (12) -- (51);
\draw (12) -- (13);
\draw (13) -- (25);
\draw (13) -- (50);
\draw (13) -- (12);
\draw (14) -- (16);
\draw (14) -- (10);
\draw (14) -- (11);
\draw (14) -- (52);
\draw (15) -- (1);
\draw (15) -- (8);
\draw (15) -- (55);
\draw (15) -- (27);
\draw (15) -- (29);
\draw (15) -- (30);
\draw (15) -- (31);
\draw (16) -- (10);
\draw (16) -- (42);
\draw (16) -- (46);
\draw (16) -- (14);
\draw (16) -- (52);
\draw (16) -- (25);
\draw (17) -- (0);
\draw (17) -- (1);
\draw (18) -- (2);
\draw (18) -- (3);
\draw (18) -- (4);
\draw (19) -- (56);
\draw (19) -- (59);
\draw (19) -- (20);
\draw (19) -- (6);
\draw (20) -- (9);
\draw (20) -- (58);
\draw (20) -- (19);
\draw (20) -- (28);
\draw (21) -- (0);
\draw (21) -- (1);
\draw (21) -- (6);
\draw (21) -- (7);
\draw (22) -- (35);
\draw (22) -- (36);
\draw (22) -- (10);
\draw (22) -- (43);
\draw (22) -- (23);
\draw (23) -- (10);
\draw (23) -- (43);
\draw (23) -- (11);
\draw (23) -- (22);
\draw (23) -- (31);
\draw (24) -- (39);
\draw (24) -- (40);
\draw (24) -- (42);
\draw (24) -- (48);
\draw (24) -- (49);
\draw (24) -- (50);
\draw (24) -- (25);
\draw (25) -- (42);
\draw (25) -- (13);
\draw (25) -- (16);
\draw (25) -- (50);
\draw (25) -- (52);
\draw (25) -- (24);
\draw (26) -- (56);
\draw (26) -- (4);
\draw (26) -- (5);
\draw (26) -- (7);
\draw (27) -- (1);
\draw (27) -- (29);
\draw (27) -- (15);
\draw (28) -- (1);
\draw (28) -- (20);
\draw (28) -- (6);
\draw (28) -- (9);
\draw (29) -- (32);
\draw (29) -- (34);
\draw (29) -- (27);
\draw (29) -- (15);
\draw (30) -- (55);
\draw (30) -- (31);
\draw (30) -- (53);
\draw (30) -- (15);
\draw (31) -- (33);
\draw (31) -- (35);
\draw (31) -- (15);
\draw (31) -- (53);
\draw (31) -- (23);
\draw (31) -- (30);
\draw (32) -- (34);
\draw (32) -- (36);
\draw (32) -- (29);
\draw (33) -- (35);
\draw (33) -- (31);
\draw (34) -- (32);
\draw (34) -- (29);
\draw (35) -- (33);
\draw (35) -- (36);
\draw (35) -- (22);
\draw (35) -- (31);
\draw (36) -- (32);
\draw (36) -- (35);
\draw (36) -- (37);
\draw (36) -- (41);
\draw (36) -- (22);
\draw (37) -- (41);
\draw (37) -- (36);
\draw (38) -- (10);
\draw (38) -- (46);
\draw (39) -- (40);
\draw (39) -- (24);
\draw (39) -- (48);
\draw (39) -- (46);
\draw (40) -- (24);
\draw (40) -- (49);
\draw (40) -- (39);
\draw (41) -- (10);
\draw (41) -- (36);
\draw (41) -- (37);
\draw (42) -- (16);
\draw (42) -- (24);
\draw (42) -- (46);
\draw (42) -- (25);
\draw (43) -- (22);
\draw (43) -- (23);
\draw (44) -- (49);
\draw (44) -- (51);
\draw (44) -- (45);
\draw (45) -- (51);
\draw (45) -- (44);
\draw (45) -- (47);
\draw (46) -- (38);
\draw (46) -- (39);
\draw (46) -- (10);
\draw (46) -- (42);
\draw (46) -- (16);
\draw (46) -- (48);
\draw (47) -- (51);
\draw (47) -- (45);
\draw (48) -- (24);
\draw (48) -- (46);
\draw (48) -- (39);
\draw (49) -- (40);
\draw (49) -- (44);
\draw (49) -- (12);
\draw (49) -- (50);
\draw (49) -- (51);
\draw (49) -- (24);
\draw (50) -- (24);
\draw (50) -- (49);
\draw (50) -- (13);
\draw (50) -- (25);
\draw (51) -- (12);
\draw (51) -- (44);
\draw (51) -- (45);
\draw (51) -- (47);
\draw (51) -- (49);
\draw (52) -- (16);
\draw (52) -- (25);
\draw (52) -- (14);
\draw (53) -- (31);
\draw (53) -- (54);
\draw (53) -- (30);
\draw (54) -- (53);
\draw (54) -- (55);
\draw (55) -- (9);
\draw (55) -- (30);
\draw (55) -- (54);
\draw (55) -- (15);
\draw (56) -- (57);
\draw (56) -- (26);
\draw (56) -- (19);
\draw (56) -- (6);
\draw (57) -- (56);
\draw (57) -- (59);
\draw (57) -- (5);
\draw (58) -- (59);
\draw (58) -- (20);
\draw (59) -- (57);
\draw (59) -- (58);
\draw (59) -- (19);
\end{tikzpicture}

\subsubsection{0\% Toleranz}
\begin{lstlisting}
$ python main.py ../../material/a3-abbiegen/abbiegen3.txt 0
Distance 17.3
6 turns
(0, 0) -> (1, 1) -> (2, 1) -> (3, 1) -> (4, 1) -> (5, 1) -> (7, 2) -> (9, 3) -> (10, 2) -> (10, 1) -> (11, 1) -> (12, 1) -> (13, 1) -> (14, 1) -> (14, 0)
\end{lstlisting}

\subsubsection{15\% Toleranz}
\begin{lstlisting}
$ python main.py ../../material/a3-abbiegen/abbiegen3.txt 15
Distance 17.3
6 turns
(0, 0) -> (1, 1) -> (2, 1) -> (3, 1) -> (4, 1) -> (5, 1) -> (7, 2) -> (9, 3) -> (10, 2) -> (10, 1) -> (11, 1) -> (12, 1) -> (13, 1) -> (14, 1) -> (14, 0)
\end{lstlisting}

\subsubsection{30\% Toleranz}
\begin{lstlisting}
$ python main.py ../../material/a3-abbiegen/abbiegen3.txt 30
Distance 17.3
6 turns
(0, 0) -> (1, 1) -> (2, 1) -> (3, 1) -> (4, 1) -> (5, 1) -> (7, 2) -> (9, 3) -> (10, 2) -> (10, 1) -> (11, 1) -> (12, 1) -> (13, 1) -> (14, 1) -> (14, 0)
\end{lstlisting}

\subsection{Eingabedatei existiert nicht}
Falls die Eingabedatei nicht existiert, wird ein Fehler zurückgegeben:
\begin{lstlisting}
$ python main.py ../../material/a3-abbiegen/abbiegen.txt 0
Traceback (most recent call last):
  File "main.py", line 169, in <module>
    junctions, roads, source, target = parse_input(sys.argv[1])
  File "main.py", line 103, in parse_input
    with open(file) as f:
FileNotFoundError: [Errno 2] No such file or directory: '../../material/a3-abbiegen/abbiegen.txt'
\end{lstlisting}

\subsection{Toleranz keine Zahl oder negativ}
Ist die Toleranz keine Zahl oder negativ, wird ein Fehler zurückgegeben:
\begin{lstlisting}
$ python main.py ../../material/a3-abbiegen/abbiegen.txt -1
Usage main.py <filename> <tolerance (percent)>
\end{lstlisting}

\subsection{Zu wenig Parameter}
Sind zu wenig Parameter angegeben, wird ein Fehler zurückgegeben:
\begin{lstlisting}
$ python main.py
Usage main.py <filename> <tolerance (percent)>
\end{lstlisting}

\subsection{Kein Weg zum Ziel}
Führt kein Weg zum Ziel, wird ein Fehler zurückgegeben.

\subsubsection{Eingabedatei}
Die Eingabedatei \texttt{error.txt}:
\begin{lstlisting}
2
(0,0)
(1,1)
(0,0) (1,0)
(0,1) (1,1)
\end{lstlisting}

\subsubsection{Visualisierung}
\begin{tikzpicture}
[auto=left,every node/.style={circle,fill=blue!20}]
\node [style={fill=green!60}] (0) at (0, 0) {0};
\node (1) at (1, 0) {1};
\node (2) at (0, 1) {2};
\node [style={fill=red!60}] (3) at (1, 1) {3};
\draw (0) -- (1);
\draw (1) -- (0);
\draw (2) -- (3);
\draw (3) -- (2);
\end{tikzpicture}

\subsubsection{Ausgabe}
Die Programmausgabe:
\begin{lstlisting}
$ python main.py error.txt 0
Error: Cannot find any path from source to target in road network!
\end{lstlisting}

\section{Quellcode (ausschnittsweise)}
Die Auslassungen belaufen sich auf die Parsen der Parameter und die Visualisierungsfunktion und sind mit \texttt{...} gekennzeichnet.
\lstinputlisting[language=Python]{tikz/code.py}
\todo{Quellcode (ausschnittsweise)}

\printbibliography

\end{document}